\documentclass[aspectratio=169,10pt]{beamer}
\usepackage{../theme}
\usepackage{colortbl}

% METADATA

\title{Fusión y Conflictos}
\subtitle{Merge, Rebase, Squash y el Arte de Sobrevivir}
\author{GLUD --- Grupo GNU/Linux Universidad Distrital}
\date{Control de Versiones y Desarrollo Colaborativo}

% DOCUMENT

\begin{document}

% Portada
\titleframe[Clase 5 · Fusión y Conflictos]

\contentsframe

% SECTION 1: INTRODUCCIÓN

\section{El Momento de la Verdad}

\begin{frame}{¿Por Qué Fusionar?}
    \vspace{0.3cm}
    
    \begin{orangebox}[El Problema]
        Después de trabajar en ramas separadas, eventualmente necesitas \highlight{integrar} esos cambios de vuelta.
    \end{orangebox}
    
    \vspace{0.4cm}
    \pause
    
    \begin{columns}[T]
        \begin{column}{0.48\textwidth}
            \textbf{Escenarios comunes:}
            \begin{itemize}
                \item Feature completada → merge a develop
                \item Hotfix urgente → merge a main
                \item Sincronizar tu rama con main
                \item Integrar trabajo de otro developer
            \end{itemize}
        \end{column}
        \pause
        \begin{column}{0.48\textwidth}
            \textbf{El miedo real:}
            \begin{itemize}
                \item ¿Y si rompo todo?
                \item ¿Qué son estos conflictos?
                \item ¿Por qué Git me odia?
                \item ¿Borro el repo y lo clono de nuevo?
            \end{itemize}
        \end{column}
    \end{columns}
\end{frame}

\begin{frame}{La ``Solución'' del Principiante}
    \vspace{0.5cm}
    
    \begin{center}
        \begin{tikzpicture}
            \node[text=gludWhite, font=\Large\bfseries] at (0,2) {Cuando hay un conflicto en Git...};
            
            \pause
            
            \node[rectangle, rounded corners, fill=gludOrange!30, text=black, font=\ttfamily\small, inner sep=15pt, text width=10cm, align=center] at (0,0) {
                rm -rf mi-proyecto/\\
                git clone https://github.com/equipo/proyecto.git\\
                \# Copiar mis cambios manualmente...\\
                \# Llorar un poco...\\
                git add . \&\& git commit -m "cosas"
            };
            
            \pause
            
            \node[text=gludOrangeLight, font=\large\itshape] at (0,-2) {``Si funciona, no es estúpido'' --- Mentira, sí lo es};
        \end{tikzpicture}
    \end{center}
\end{frame}

\quoteslide{Merge conflicts are not bugs --- they are features that tell you two people cared enough to change the same thing.}{Sabiduría de Stack Overflow}

\begin{frame}{Las Herramientas de Fusión}
    \vspace{0.3cm}
    
    \textbf{Git ofrece varias formas de integrar cambios:}
    
    \vspace{0.4cm}
    
    \begin{columns}[T]
        \begin{column}{0.32\textwidth}
            \begin{darkbox}[git merge]
                {\color{gludWhite}
                \begin{itemize}
                    \item Combina historiales
                    \item Preserva contexto
                    \item Crea commit de merge
                    \item El más común
                \end{itemize}
                }
            \end{darkbox}
        \end{column}
        \pause
        \begin{column}{0.32\textwidth}
            \begin{darkbox}[git rebase]
                {\color{gludWhite}
                \begin{itemize}
                    \item Reescribe historial
                    \item Historial lineal
                    \item Más limpio
                    \item Más peligroso
                \end{itemize}
                }
            \end{darkbox}
        \end{column}
        \pause
        \begin{column}{0.32\textwidth}
            \begin{darkbox}[git squash]
                {\color{gludWhite}
                \begin{itemize}
                    \item Combina commits
                    \item Un solo commit
                    \item Historial simple
                    \item Pierde detalles
                \end{itemize}
                }
            \end{darkbox}
        \end{column}
    \end{columns}
\end{frame}

% SECTION 2: GIT MERGE

\section{Git Merge}

\begin{frame}{¿Qué es git merge?}
    \begin{orangebox}[Definición]
        \texttt{git merge} combina el historial de dos ramas, integrando los cambios de una rama en otra.
    \end{orangebox}
    
    \pause
    
    \begin{codebox}[Sintaxis básica]
        \ttfamily\color{gludWhite}
        git switch main\\
        git merge feature/login
    \end{codebox}
    
    {\color{gludLightGray} Esto trae los cambios de \texttt{feature/login} hacia \texttt{main}}

    \pause
    
    \begin{alertbox}[Importante]
        {\color{black}Siempre estás \textbf{trayendo} cambios \textbf{hacia} la rama en la que te encuentras.}
    \end{alertbox}
\end{frame}

\begin{frame}{Tipos de Merge}
    \vspace{0.3cm}
    
    Git decide automáticamente qué tipo de merge hacer:
    
    \vspace{0.4cm}
    
    \begin{columns}[T]
        \begin{column}{0.48\textwidth}
            \begin{darkbox}[Fast-Forward]
                {\color{gludWhite}
                \textbf{Cuándo ocurre:}\\
                La rama base no tiene commits nuevos desde que se creó la feature.
                
                \vspace{0.3cm}
                
                \textbf{Resultado:}\\
                Git simplemente ``avanza'' el puntero. No crea commit de merge.
                }
            \end{darkbox}
        \end{column}
        \pause
        \begin{column}{0.48\textwidth}
            \begin{darkbox}[3-Way Merge (Recursive)]
                {\color{gludWhite}
                \textbf{Cuándo ocurre:}\\
                Ambas ramas tienen commits nuevos (historiales divergentes).
                
                \vspace{0.3cm}
                
                \textbf{Resultado:}\\
                Git crea un nuevo ``merge commit'' con dos padres.
                }
            \end{darkbox}
        \end{column}
    \end{columns}
\end{frame}

\begin{frame}{Fast-Forward Merge: Visualización}
    \vspace{0.3cm}
    
    \begin{center}
        \textbf{ANTES del merge:}
        \vspace{0.3cm}
        
        \begin{tikzpicture}[
            commit/.style={circle, fill=gludOrange, text=white, font=\tiny\bfseries, minimum size=0.5cm},
            branch/.style={rectangle, rounded corners, fill=gludDarker, text=gludOrange, font=\tiny\ttfamily, inner sep=2pt},
            arrow/.style={->, thick, gludOrangeLight}
        ]
            \node[commit] (c1) at (0,0) {C1};
            \node[commit] (c2) at (1.5,0) {C2};
            \node[commit] (c3) at (3,0) {C3};
            \node[commit] (c4) at (4.5,0) {C4};
            
            \draw[arrow] (c1) -- (c2);
            \draw[arrow] (c2) -- (c3);
            \draw[arrow] (c3) -- (c4);
            
            \node[branch] at (1.5,-0.7) {main};
            \node[branch] at (4.5,-0.7) {feature};
        \end{tikzpicture}
    \end{center}
    
    \pause
    
    \begin{center}
        \textbf{DESPUÉS de \texttt{git merge feature}:}
        \vspace{0.3cm}
        
        \begin{tikzpicture}[
            commit/.style={circle, fill=gludOrange, text=white, font=\tiny\bfseries, minimum size=0.5cm},
            branch/.style={rectangle, rounded corners, fill=gludDarker, text=gludOrange, font=\tiny\ttfamily, inner sep=2pt},
            arrow/.style={->, thick, gludOrangeLight}
        ]
            \node[commit] (c1) at (0,0) {C1};
            \node[commit] (c2) at (1.5,0) {C2};
            \node[commit] (c3) at (3,0) {C3};
            \node[commit] (c4) at (4.5,0) {C4};
            
            \draw[arrow] (c1) -- (c2);
            \draw[arrow] (c2) -- (c3);
            \draw[arrow] (c3) -- (c4);
            
            \node[branch] at (4.5,-0.7) {main, feature};
        \end{tikzpicture}
    \end{center}
    
    \vspace{0.3cm}
    {\color{gludLightGray}\scriptsize main simplemente ``avanzó'' hasta donde estaba feature. No hay commit nuevo.}
\end{frame}

\begin{frame}{3-Way Merge: Visualización}
    \vspace{0.2cm}
    \begin{columns}[T]
        \begin{column}{0.48\textwidth}
            \centering
            \textbf{ANTES del merge (historiales divergentes):}
            \vspace{0.3cm}
            
            \begin{tikzpicture}[
                commit/.style={circle, fill=gludOrange, text=white, font=\tiny\bfseries, minimum size=0.5cm},
                branch/.style={rectangle, rounded corners, fill=gludDarker, text=gludOrange, font=\tiny\ttfamily, inner sep=2pt},
                arrow/.style={->, thick, gludOrangeLight}
            ]
                \node[commit] (c1) at (0,0) {C1};
                \node[commit] (c2) at (1.5,0) {C2};
                \node[commit] (c3) at (3,1) {C3};
                \node[commit] (c4) at (3,-1) {C4};
                \node[commit] (c5) at (4.5,-1) {C5};
                
                \draw[arrow] (c1) -- (c2);
                \draw[arrow] (c2) -- (c3);
                \draw[arrow] (c2) -- (c4);
                \draw[arrow] (c4) -- (c5);
                
                \node[branch] at (3,1.6) {main};
                \node[branch] at (4.5,-1.6) {feature};
            \end{tikzpicture}
        \end{column}
        \begin{column}{0.48\textwidth}
            \centering
            \textbf{DESPUÉS de \texttt{git merge feature}:}
            \vspace{0.3cm}
            
            \begin{tikzpicture}[
                commit/.style={circle, fill=gludOrange, text=white, font=\tiny\bfseries, minimum size=0.5cm},
                mergecommit/.style={circle, fill=gludPeach, text=gludDark, font=\tiny\bfseries, minimum size=0.5cm},
                branch/.style={rectangle, rounded corners, fill=gludDarker, text=gludOrange, font=\tiny\ttfamily, inner sep=2pt},
                arrow/.style={->, thick, gludOrangeLight}
            ]
                \node[commit] (c1) at (0,0) {C1};
                \node[commit] (c2) at (1.5,0) {C2};
                \node[commit] (c3) at (3,1) {C3};
                \node[commit] (c4) at (3,-1) {C4};
                \node[commit] (c5) at (4.5,-1) {C5};
                \node[mergecommit] (m) at (5.5,0) {M};
                
                \draw[arrow] (c1) -- (c2);
                \draw[arrow] (c2) -- (c3);
                \draw[arrow] (c2) -- (c4);
                \draw[arrow] (c4) -- (c5);
                \draw[arrow] (c3) -- (m);
                \draw[arrow] (c5) -- (m);
                
                \node[branch] at (5.5,0.7) {main};
                \node[branch] at (4.5,-1.6) {feature};
            \end{tikzpicture}
        \end{column}
    \end{columns}
    
    \vspace{0.2cm}
    {\color{gludLightGray}\scriptsize Git crea un ``merge commit'' (M) que tiene DOS padres: C3 y C5}
\end{frame}

\begin{frame}{Forzar el Tipo de Merge}
    \vspace{0.3cm}
    
    \begin{columns}[T]
        \begin{column}{0.48\textwidth}
            \begin{codebox}[Forzar merge commit]
                \ttfamily\small\color{gludWhite}
                git merge -{-}no-ff feature
            \end{codebox}
            
            \vspace{0.3cm}
            
            {\small\color{gludLightGray} Aunque sea posible fast-forward, crea un commit de merge. Útil para mantener registro de qué fue una feature.}
        \end{column}
        \pause
        \begin{column}{0.48\textwidth}
            \begin{codebox}[Forzar fast-forward]
                \ttfamily\small\color{gludWhite}
                git merge -{-}ff-only feature
            \end{codebox}
            
            \vspace{0.3cm}
            
            {\small\color{gludLightGray} Solo hace merge si es fast-forward. Falla si hay divergencia. Útil para scripts de CI.}
        \end{column}
    \end{columns}
    
    \vspace{0.5cm}
    \pause
    
    \begin{orangebox}[Recomendación]
        {\color{black}En Git Flow, usa \texttt{-{-}no-ff} para features. Así puedes ver en el historial dónde empezó y terminó cada feature.}
    \end{orangebox}
\end{frame}

\begin{frame}[fragile]{Ejemplo Práctico: Merge Exitoso}
    \vspace{0.3cm}
    
    \begin{codebox}[Flujo completo de merge]
        \ttfamily\small\color{gludWhite}
        \# Asegurarse de estar en main actualizado\\
        git switch main\\
        git pull origin main\\
        \\
        \# Hacer el merge\\
        git merge feature/login\\
        \\
        \# Si todo sale bien...\\
        git push origin main\\
        \\
        \# Limpiar la rama\\
        git branch -d feature/login
    \end{codebox}
    
    \vspace{0.3cm}
    
    \begin{codebox}[Mensaje típico de éxito]
        \ttfamily\tiny\color{gludWhite}
        Merge made by the 'ort' strategy.\\
        \ login.py | 45 +++++++++++++++++++++\\
        \ 1 file changed, 45 insertions(+)
    \end{codebox}
\end{frame}

% SECTION 3: GIT REBASE

\section{Git Rebase}

\begin{frame}{¿Qué es git rebase?}
    \vspace{0.3cm}
    
    \begin{orangebox}[Definición]
        \texttt{git rebase} \highlight{reescribe} el historial de commits, moviendo tu rama a una nueva base (el tip de otra rama).
    \end{orangebox}
    
    \vspace{0.4cm}
    \pause
    
    \begin{columns}[T]
        \begin{column}{0.48\textwidth}
            \textbf{Concepto:}
            \begin{itemize}
                \item ``Re-basar'' = cambiar la base
                \item Toma tus commits
                \item Los ``replanta'' sobre otro punto
                \item Crea nuevos commits (nuevos hashes)
            \end{itemize}
        \end{column}
        \pause
        \begin{column}{0.48\textwidth}
            \textbf{Resultado:}
            \begin{itemize}
                \item Historial lineal
                \item Sin commits de merge
                \item Más limpio visualmente
                \item Pero... reescribe historial
            \end{itemize}
        \end{column}
    \end{columns}
\end{frame}

\begin{frame}{Rebase: Visualización}
    \vspace{0.2cm}
    
    \begin{center}
        \textbf{ANTES del rebase:}
        \vspace{0.2cm}
        
        \begin{tikzpicture}[
            commit/.style={circle, fill=gludOrange, text=white, font=\tiny\bfseries, minimum size=0.5cm},
            branch/.style={rectangle, rounded corners, fill=gludDarker, text=gludOrange, font=\tiny\ttfamily, inner sep=2pt},
            arrow/.style={->, thick, gludOrangeLight}
        ]
            \node[commit] (c1) at (0,0) {C1};
            \node[commit] (c2) at (1.5,0) {C2};
            \node[commit] (c3) at (3,0) {C3};
            \node[commit] (c4) at (4.5,0) {C4};
            \node[commit] (f1) at (3,1.5) {F1};
            \node[commit] (f2) at (4.5,1.5) {F2};
            
            \draw[arrow] (c1) -- (c2);
            \draw[arrow] (c2) -- (c3);
            \draw[arrow] (c3) -- (c4);
            \draw[arrow] (c2) -- (f1);
            \draw[arrow] (f1) -- (f2);
            
            \node[branch] at (4.5,-0.7) {main};
            \node[branch] at (4.5,2.2) {feature};
        \end{tikzpicture}
    \end{center}
    
    \pause
    
    \begin{center}
        \textbf{DESPUÉS de \texttt{git rebase main} (estando en feature):}
        \vspace{0.2cm}
        
        \begin{tikzpicture}[
            commit/.style={circle, fill=gludOrange, text=white, font=\tiny\bfseries, minimum size=0.5cm},
            newcommit/.style={circle, fill=gludPeach, text=gludDark, font=\tiny\bfseries, minimum size=0.5cm},
            branch/.style={rectangle, rounded corners, fill=gludDarker, text=gludOrange, font=\tiny\ttfamily, inner sep=2pt},
            arrow/.style={->, thick, gludOrangeLight}
        ]
            \node[commit] (c1) at (0,0) {C1};
            \node[commit] (c2) at (1.5,0) {C2};
            \node[commit] (c3) at (3,0) {C3};
            \node[commit] (c4) at (4.5,0) {C4};
            \node[newcommit] (f1) at (6,0) {F1'};
            \node[newcommit] (f2) at (7.5,0) {F2'};
            
            \draw[arrow] (c1) -- (c2);
            \draw[arrow] (c2) -- (c3);
            \draw[arrow] (c3) -- (c4);
            \draw[arrow] (c4) -- (f1);
            \draw[arrow] (f1) -- (f2);
            
            \node[branch] at (4.5,-0.7) {main};
            \node[branch] at (7.5,-0.7) {feature};
        \end{tikzpicture}
    \end{center}
    
    \vspace{0.2cm}
    {\color{gludLightGray}\scriptsize Los commits F1 y F2 ahora son F1' y F2' --- ¡son commits NUEVOS con hashes diferentes!}
\end{frame}

\begin{frame}{Cuándo Usar Rebase}
    \vspace{0.3cm}
    
    \begin{columns}[T]
        \begin{column}{0.48\textwidth}
            \textbf{\color{gludOrange} Usar rebase:}
            \begin{itemize}
                \item Actualizar tu feature con cambios de main
                \item Antes de crear un PR (historial limpio)
                \item En ramas locales no compartidas
                \item Para limpiar historial personal
            \end{itemize}
        \end{column}
        \pause
        \begin{column}{0.48\textwidth}
            \textbf{\color{gludOrange} NO usar rebase:}
            \begin{itemize}
                \item En ramas públicas/compartidas
                \item Después de hacer push
                \item En main o develop
                \item Si no entiendes las consecuencias
            \end{itemize}
        \end{column}
    \end{columns}
    
    \vspace{0.5cm}
    \pause
    
    \begin{alertbox}[La Regla de Oro del Rebase]
        {\color{black}\textbf{NUNCA hagas rebase de commits que ya hayas pusheado.} Otros pueden tener esos commits y causarás un caos dimensional.}
    \end{alertbox}
\end{frame}

\begin{frame}[fragile]{Ejemplo Práctico: Rebase}
    \begin{codebox}[Actualizar feature con cambios de main]
        \ttfamily\small\color{gludWhite}
        \# Estás en tu feature, main tiene cambios nuevos\\
        git switch feature/login\\
        \\
        \# Traer cambios de main\\
        git fetch origin main\\
        \\
        \# Rebasar tu feature sobre main\\
        git rebase origin/main\\
        \\
        \# Si hay conflictos, resolverlos y:\\
        git rebase -{-}continue\\
        \\
        \# Si todo sale mal:\\
        git rebase -{-}abort
    \end{codebox}
\end{frame}

\quoteslide{Do not rebase commits that exist outside your repository and that people may have based work on.}{Pro Git Book}

% SECTION 4: SQUASH

\section{Squash: Aplastar Commits}

\begin{frame}{¿Qué es Squash?}
    \vspace{0.3cm}
    
    \begin{orangebox}[Definición]
        \texttt{squash} combina múltiples commits en uno solo, simplificando el historial.
    \end{orangebox}
    
    \vspace{0.4cm}
    \pause
    
    \textbf{¿Por qué aplastar commits?}
    
    \vspace{0.3cm}
    
    \begin{columns}[T]
        \begin{column}{0.48\textwidth}
            \textbf{Tu historial real:}
            \begin{itemize}
                \item ``WIP''
                \item ``arreglo typo''
                \item ``ahora sí''
                \item ``por qué no funciona''
                \item ``FUNCIONA!!!''
                \item ``fix lint''
            \end{itemize}
        \end{column}
        \pause
        \begin{column}{0.48\textwidth}
            \textbf{Lo que quieres en main:}
            \begin{itemize}
                \item ``feat: implementar sistema de login con OAuth2''
            \end{itemize}
            
            \vspace{0.5cm}
            
            {\color{gludOrangeLight} Un commit limpio y descriptivo}
        \end{column}
    \end{columns}
\end{frame}

\begin{frame}{Squash: Visualización}
    \vspace{0.3cm}
    
    \begin{center}
        \textbf{ANTES (6 commits):}
        \vspace{0.3cm}
        
        \begin{tikzpicture}[
            commit/.style={circle, fill=gludOrange, text=white, font=\tiny\bfseries, minimum size=0.5cm},
            branch/.style={rectangle, rounded corners, fill=gludDarker, text=gludOrange, font=\tiny\ttfamily, inner sep=2pt},
            arrow/.style={->, thick, gludOrangeLight}
        ]
            \node[commit] (c1) at (0,0) {C1};
            \node[commit] (c2) at (1.2,0) {C2};
            \node[commit] (c3) at (2.4,0) {C3};
            \node[commit] (c4) at (3.6,0) {C4};
            \node[commit] (c5) at (4.8,0) {C5};
            \node[commit] (c6) at (6,0) {C6};
            
            \draw[arrow] (c1) -- (c2);
            \draw[arrow] (c2) -- (c3);
            \draw[arrow] (c3) -- (c4);
            \draw[arrow] (c4) -- (c5);
            \draw[arrow] (c5) -- (c6);
            
            \node[text=gludLightGray, font=\tiny] at (1.2,-0.7) {wip};
            \node[text=gludLightGray, font=\tiny] at (2.4,-0.7) {typo};
            \node[text=gludLightGray, font=\tiny] at (3.6,-0.7) {fix};
            \node[text=gludLightGray, font=\tiny] at (4.8,-0.7) {wip};
            \node[text=gludLightGray, font=\tiny] at (6,-0.7) {done};
        \end{tikzpicture}
    \end{center}
    
    \pause
    
    \begin{center}
        \textbf{DESPUÉS del squash (1 commit):}
        \vspace{0.3cm}
        
        \begin{tikzpicture}[
            commit/.style={circle, fill=gludOrange, text=white, font=\tiny\bfseries, minimum size=0.5cm},
            squashed/.style={circle, fill=gludPeach, text=gludDark, font=\tiny\bfseries, minimum size=0.7cm},
            branch/.style={rectangle, rounded corners, fill=gludDarker, text=gludOrange, font=\tiny\ttfamily, inner sep=2pt},
            arrow/.style={->, thick, gludOrangeLight}
        ]
            \node[commit] (c1) at (0,0) {C1};
            \node[squashed] (s) at (2,0) {S};
            
            \draw[arrow] (c1) -- (s);
            
            \node[text=gludLightGray, font=\tiny, text width=3cm, align=center] at (2,-0.9) {feat: login con OAuth2};
        \end{tikzpicture}
    \end{center}
    
    \vspace{0.3cm}
    {\color{gludLightGray}\scriptsize Todos los commits intermedios se combinan en uno solo con un mensaje limpio}
\end{frame}

\begin{frame}{Formas de Hacer Squash}    
    \step{1}{Squash en merge (GitHub/GitLab)}
    
    \begin{codebox}
        \ttfamily\small\color{gludWhite}
        \# En la interfaz web, botón "Squash and Merge"
    \end{codebox}
    
    \pause
    
    \step{2}{Squash con rebase interactivo}
    
    \begin{codebox}
        \ttfamily\small\color{gludWhite}
        git rebase -i HEAD\textasciitilde5  \# Últimos 5 commits
    \end{codebox}
    
    \pause
    
    \step{3}{Merge con squash}
    
    \begin{codebox}
        \ttfamily\small\color{gludWhite}
        git merge -{-}squash feature/login\\
        git commit -m "feat: implementar login"
    \end{codebox}
\end{frame}

\begin{frame}[fragile]{Rebase Interactivo: El Editor}
    Al ejecutar \texttt{git rebase -i HEAD\textasciitilde5}, Git abre un editor:

    \begin{codebox}[Editor de rebase interactivo]
        \ttfamily\tiny\color{gludWhite}
        pick a1b2c3d feat: inicio login\\
        pick b2c3d4e wip\\
        pick c3d4e5f arreglo typo\\
        pick d4e5f6g ahora sí funciona\\
        pick e5f6g7h fix lint\\
        \\
        \# Comandos:\\
        \# p, pick = usar commit\\
        \# s, squash = usar commit, pero fusionar con el anterior\\
        \# f, fixup = como squash, pero descartar mensaje\\
        \# d, drop = eliminar commit
    \end{codebox}
    
    \pause
    
    \begin{codebox}[Cambiar a:]
        \ttfamily\tiny\color{gludWhite}
        pick a1b2c3d feat: inicio login\\
        squash b2c3d4e wip\\
        squash c3d4e5f arreglo typo\\
        squash d4e5f6g ahora sí funciona\\
        squash e5f6g7h fix lint
    \end{codebox}
\end{frame}

% SECTION 5: COMPARACIÓN

\section{Merge vs Rebase vs Squash}

\begin{frame}{Tabla Comparativa}
    \vspace{0.3cm}
    
    \centering
    \small
    \renewcommand{\arraystretch}{1.4}
    \begin{tabular}{|l|c|c|c|}
        \hline
        \rowcolor{gludOrange}
        \textcolor{white}{\textbf{Aspecto}} & \textcolor{white}{\textbf{Merge}} & \textcolor{white}{\textbf{Rebase}} & \textcolor{white}{\textbf{Squash}} \\
        \hline
        \rowcolor{gludDarker}
        Historial & Preservado & Reescrito & Simplificado \\
        \hline
        \rowcolor{gludDark}
        Commits nuevos & Merge commit & Nuevos hashes & Un commit \\
        \hline
        \rowcolor{gludDarker}
        Linealidad & Ramificado & Lineal & Lineal \\
        \hline
        \rowcolor{gludDark}
        Conflictos & Una vez & Por cada commit & Una vez \\
        \hline
        \rowcolor{gludDarker}
        Seguridad & Alto & Bajo & Medio \\
        \hline
        \rowcolor{gludDark}
        Trazabilidad & Completa & Perdida & Perdida \\
        \hline
    \end{tabular}
\end{frame}

\begin{frame}{¿Cuál Usar?}
    
    \begin{columns}[T]
        \begin{column}{0.32\textwidth}
            \begin{darkbox}[Usa Merge]
                {\color{gludWhite}
                \begin{itemize}
                    \item Ramas públicas
                    \item Cuando la historia importa
                    \item En main/develop
                    \item Con -{-}no-ff en Git Flow
                \end{itemize}
                }
            \end{darkbox}
        \end{column}
        \begin{column}{0.32\textwidth}
            \begin{darkbox}[Usa Rebase]
                {\color{gludWhite}
                \begin{itemize}
                    \item Ramas locales
                    \item Antes de PR
                    \item Para historial limpio
                    \item Actualizar feature
                \end{itemize}
                }
            \end{darkbox}
        \end{column}
        \begin{column}{0.32\textwidth}
            \begin{darkbox}[Usa Squash]
                {\color{gludWhite}
                \begin{itemize}
                    \item Features pequeñas
                    \item Muchos commits WIP
                    \item PR final a main
                    \item Limpieza de historial
                \end{itemize}
                }
            \end{darkbox}
        \end{column}
    \end{columns}
    \pause
    
    \begin{orangebox}[Estrategia común de equipos]
        {\color{black}\texttt{rebase} para actualizar feature branches + \texttt{squash merge} al fusionar PR = Historial lineal y limpio en main}
    \end{orangebox}
\end{frame}

% SECTION 6: CONFLICTOS

\section{Resolución de Conflictos}

\begin{frame}{¿Qué es un Conflicto?}
    \vspace{0.3cm}
    
    \begin{orangebox}[Definición]
        Un \highlight{conflicto} ocurre cuando Git no puede determinar automáticamente cómo combinar cambios en las mismas líneas de un archivo.
    \end{orangebox}
    
    \vspace{0.4cm}
    \pause
    
    \textbf{Git puede resolver automáticamente:}
    \begin{itemize}
        \item Cambios en archivos diferentes
        \item Cambios en líneas diferentes del mismo archivo
        \item Agregar contenido nuevo sin solapar
    \end{itemize}
    
    \pause
    
    \textbf{Git NO puede resolver:}
    \begin{itemize}
        \item Cambios en las MISMAS líneas por diferentes personas
        \item Eliminación de un archivo que otro modificó
        \item Cambios que dependen lógicamente entre sí
    \end{itemize}
\end{frame}

\begin{frame}[fragile]{Anatomía de un Conflicto}
    
    \begin{codebox}[Así se ve un conflicto en el archivo]
        \ttfamily\small\color{gludWhite}
        def calcular\_precio(cantidad):\\
        <<<<<<< HEAD\\
        \hspace*{0.5cm}return cantidad * 100  \# Tu cambio\\
        =======\\
        \hspace*{0.5cm}return cantidad * 150  \# Cambio del otro\\
        >>>>>>> feature/precios\\
    \end{codebox}
    \pause
    
    \begin{columns}[T]
        \begin{column}{0.48\textwidth}
            \textbf{Marcadores:}
            \begin{itemize}
                \item \texttt{<<<<<<< HEAD} --- Inicio de tu versión
                \item \texttt{=======} --- Separador
                \item \texttt{>>>>>>>} --- Fin de la otra versión
            \end{itemize}
        \end{column}
        \begin{column}{0.48\textwidth}
            \textbf{Tu trabajo:}
            \begin{itemize}
                \item Decidir qué versión conservar
                \item O combinar ambas manualmente
                \item Eliminar los marcadores
            \end{itemize}
        \end{column}
    \end{columns}
\end{frame}

\begin{frame}[fragile]{Resolviendo el Conflicto}
    \vspace{0.3cm}
    
    \step{1}{Identificar archivos con conflictos}
    
    \begin{codebox}
        \ttfamily\small\color{gludWhite}
        git status {\small\color{gludLightGray}\# Muestra "both modified"}
    \end{codebox}
    
    \pause
    
    \step{2}{Editar el archivo y elegir la solución}
    
    \begin{codebox}[Archivo resuelto]
        \ttfamily\small\color{gludWhite}
        def calcular\_precio(cantidad, premium=False):\\
        \ \ \ \ precio\_base = 100\\
        \ \ \ \ if premium:\\
        \ \ \ \ \ \ \ \ return cantidad * 150\\
        \ \ \ \ return cantidad * precio\_base
    \end{codebox}
\end{frame}

\begin{frame}[fragile]{Resolviendo el Conflicto (cont.)}
    \vspace{0.3cm}
    
    \step{3}{Marcar como resuelto y continuar}
    
    \begin{codebox}
        \ttfamily\small\color{gludWhite}
        git add precio.py\\
        git commit -m "fix: resolver conflicto de precios"
    \end{codebox}
\end{frame}

\begin{frame}{Herramientas para Resolver Conflictos}
    \vspace{0.3cm}
    
    \begin{columns}[T]
        \begin{column}{0.48\textwidth}
            \begin{darkbox}[VS Code]
                {\color{gludWhite}
                \begin{itemize}
                    \item Vista de 3 paneles
                    \item Botones: Accept Incoming/Current/Both
                    \item Integración con GitLens
                    \item ¡Lo más fácil!
                \end{itemize}
                }
            \end{darkbox}
        \end{column}
        \begin{column}{0.48\textwidth}
            \begin{darkbox}[Terminal]
                {\color{gludWhite}
                \begin{codebox}
                    \ttfamily\tiny\color{gludWhite}
                    \# Usar herramienta visual\\
                    git mergetool\\
                    \\
                    \# Abortar merge\\
                    git merge -{-}abort\\
                    \\
                    \# Ver versiones\\
                    git checkout -{-}ours file\\
                    git checkout -{-}theirs file
                \end{codebox}
                }
            \end{darkbox}
        \end{column}
    \end{columns}
    
    \pause
    
    \begin{orangebox}[Consejo]
        {\color{black}Configura VS Code como tu merge tool: \texttt{git config -{-}global merge.tool vscode}}
    \end{orangebox}
\end{frame}

\begin{frame}{Ours vs Theirs}
    
    \begin{alertbox}[¡Cuidado!]
        El significado cambia según la operación
    \end{alertbox}
    
    \begin{columns}[T]
        \begin{column}{0.48\textwidth}
            \begin{darkbox}[Durante MERGE]
                {\color{gludWhite}
                \texttt{-{-}ours} = rama donde estás (main)\\
                \texttt{-{-}theirs} = rama que fusionas (feature)
                
                \begin{codebox}
                    \ttfamily\tiny\color{gludWhite}
                    git switch main\\
                    git merge feature\\
                    \# ours = main\\
                    \# theirs = feature
                \end{codebox}
                }
            \end{darkbox}
        \end{column}
        \pause
        \begin{column}{0.48\textwidth}
            \begin{darkbox}[Durante REBASE]
                {\color{gludWhite}
                \texttt{-{-}ours} = rama donde rebaseas (main)\\
                \texttt{-{-}theirs} = tus commits (feature)
                
                \vspace{0.3cm}
                
                \begin{codebox}
                    \ttfamily\tiny\color{gludWhite}
                    git switch feature\\
                    git rebase main\\
                    \# ours = main (!)\\
                    \# theirs = feature (!)
                \end{codebox}
                }
            \end{darkbox}
        \end{column}
    \end{columns}
    
    \vspace{0.3cm}
    
    {\color{gludLightGray}\scriptsize Sí, es confuso. En rebase, Git está ``replicando'' tus commits sobre main, así que main es ``ours'' temporalmente.}
\end{frame}

\begin{frame}{Tips de Supervivencia}
    \vspace{0.3cm}
    
    \begin{columns}[T]
        \begin{column}{0.48\textwidth}
            \textbf{\color{gludOrange} Antes del merge:}
            \begin{itemize}
                \item \texttt{git stash} si tienes cambios sin commit
                \item \texttt{git pull} para tener todo actualizado
                \item Revisar \texttt{git log} de ambas ramas
                \item Hacer backup mental del estado
            \end{itemize}
        \end{column}
        \pause
        \begin{column}{0.48\textwidth}
            \textbf{\color{gludOrange} Durante el conflicto:}
            \begin{itemize}
                \item Mantener la calma
                \item Resolver un archivo a la vez
                \item Probar el código después
                \item Pedir ayuda si es necesario
            \end{itemize}
        \end{column}
    \end{columns}
    
    \vspace{0.5cm}
    \pause
    
    \begin{codebox}[Comandos de emergencia]
        \ttfamily\small\color{gludWhite}
        git merge -{-}abort\ \ \ \ \# Cancelar merge en progreso\\
        git reset -{-}hard HEAD\ \# Deshacer TODO (peligroso)\\
        git reflog\ \ \ \ \ \ \ \ \ \ \ \# Ver historial completo
    \end{codebox}
\end{frame}

\begin{frame}{Lo que NO Debes Hacer}
    \vspace{0.3cm}
    
    \begin{center}
        \begin{tikzpicture}
            \node[rectangle, rounded corners, fill=gludOrange!20, text=black, inner sep=15pt, text width=12cm] at (0,0) {
                \textbf{\Large Las ``Soluciones'' que No Son Soluciones:}
                
                \vspace{0.3cm}
                
                \begin{itemize}
                    {\color{black}
                    \item[$\times$] Borrar el repositorio y clonarlo de nuevo
                    \item[$\times$] Copiar los archivos a otra carpeta y empezar de cero
                    \item[$\times$] Hacer \texttt{git checkout -{-}theirs .} sin revisar
                    \item[$\times$] Editar el archivo remoto directamente en GitHub
                    \item[$\times$] Crear una rama ``backup'' cada vez que hay conflicto
                    \item[$\times$] Ignorar los marcadores de conflicto y hacer push
                    }
                \end{itemize}
            };
        \end{tikzpicture}
    \end{center}
    
    \pause
    
    \vspace{0.3cm}
    
    {\color{gludLightGray}\centering\small Todos hemos hecho al menos una de estas... pero hoy aprendemos a hacerlo bien.}
\end{frame}

% SECTION 8: RESUMEN

\section{Resumen y Próximos Pasos}

\begin{frame}{Resumen de la Clase}
    \vspace{0.3cm}
    
    \begin{columns}[T]
        \begin{column}{0.48\textwidth}
            \textbf{\color{gludOrange} Conceptos clave:}
            \begin{itemize}
                \item \textbf{Merge:} Combina historiales, preserva contexto
                \item \textbf{Rebase:} Reescribe historial, mantiene linealidad
                \item \textbf{Squash:} Combina commits en uno
                \item \textbf{Conflictos:} Git necesita tu ayuda
            \end{itemize}
        \end{column}
        \pause
        \begin{column}{0.48\textwidth}
            \textbf{\color{gludOrange} Reglas importantes:}
            \begin{itemize}
                \item Nunca rebase commits pusheados
                \item Merge para ramas públicas
                \item Resolver conflictos uno a uno
                \item \texttt{git reflog} es tu amigo
            \end{itemize}
        \end{column}
    \end{columns}
\end{frame}

\begin{frame}{Comandos Clave}

    \begin{columns}[T]
        \begin{column}{0.48\textwidth}
            \begin{darkbox}[Merge]
                {\color{gludWhite}
                \begin{codebox}
                    \ttfamily\tiny\color{gludWhite}
                    git merge feature\\
                    git merge -{-}no-ff feature\\
                    git merge -{-}squash feature\\
                    git merge -{-}abort
                \end{codebox}
                }
            \end{darkbox}
            
            \vspace{0.1cm}
            
            \begin{darkbox}[Rebase]
                {\color{gludWhite}
                \begin{codebox}
                    \ttfamily\tiny\color{gludWhite}
                    git rebase main\\
                    git rebase -i HEAD\textasciitilde5\\
                    git rebase -{-}continue\\
                    git rebase -{-}abort
                \end{codebox}
                }
            \end{darkbox}
        \end{column}
        \begin{column}{0.48\textwidth}
            \begin{darkbox}[Conflictos]
                {\color{gludWhite}
                \begin{codebox}
                    \ttfamily\tiny\color{gludWhite}
                    git status\\
                    git diff\\
                    git add archivo\\
                    git checkout -{-}ours file\\
                    git checkout -{-}theirs file
                \end{codebox}
                }
            \end{darkbox}
            
            
            \begin{darkbox}[Emergencias]
                {\color{gludWhite}
                \begin{codebox}
                    \ttfamily\tiny\color{gludWhite}
                    git reflog\\
                    git reset -{-}hard HASH\\
                    git stash\\
                    git stash pop
                \end{codebox}
                }
            \end{darkbox}
        \end{column}
    \end{columns}
\end{frame}

\begin{frame}{Próxima Clase: Flujos Remotos}
    \vspace{0.3cm}
    
    \begin{orangebox}[Clase 6]
        {\color{black}Hasta ahora hemos trabajado localmente. Es hora de conectar con el mundo exterior: \highlight{GitHub y GitLab}.}
    \end{orangebox}
    
    \vspace{0.4cm}
    
    \textbf{En la Clase 6 aprenderemos:}
    \begin{itemize}
        \item Autenticación SSH (clave pública/privada)
        \item Conceptos de \texttt{origin} y \texttt{upstream}
        \item Forks y contribución a proyectos ajenos
        \item Configurar repositorios remotos
        \item Flujos de trabajo y gestion de ramas y repositorio
    \end{itemize}
    
    \vspace{0.4cm}
    
    \centering
    {\color{gludOrangeLight} ¡Preparen sus cuentas de GitHub/GitLab!}
\end{frame}

\quoteslide{The only way to learn Git is to use Git --- especially when things go wrong.}{Scott Chacon, Pro Git}

\thankslide{¡Gracias!\\[0.3cm]{\hspace*{1cm}\Large Sobrevivieron al Laboratorio de Caos}}

\end{document}
