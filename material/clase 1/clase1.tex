\documentclass[aspectratio=169,10pt]{beamer}
\usepackage{../theme}

% METADATA

\title{Configuración del Entorno}
\subtitle{Git + Java + Maven en Linux}
\author{GLUD --- Grupo GNU/Linux Universidad Distrital}
\date{Control de Versiones y Desarrollo Colaborativo}

% DOCUMENT

\begin{document}

% Portada
\titleframe[Clase 1 · Primeros pasos]

\contentsframe

% SECTION 1: INTRODUCCIÓN

\section{¿Por qué Git?}

\begin{frame}{El valor de Git en tu flujo de trabajo}
    \vspace{0.3cm}
    \begin{columns}[T]
        \begin{column}{0.48\textwidth}
            \begin{orangebox}[Sin Git]
                \begin{itemize}
                    \item<1-> {\color{black}Archivos duplicados sin control}
                    \item<2-> {\color{black}Colaboración por USB o email}
                    \item<3-> {\color{black}Miedo a experimentar}
                    \item<4-> {\color{black}No hay registro de cambios}
                \end{itemize}
            \end{orangebox}
        \end{column}
        \pause
        \begin{column}{0.48\textwidth}
            \begin{darkbox}[Con Git]
                \begin{itemize}
                    \item<2-> Historial completo y navegable
                    \item<3-> Colaboración simultánea
                    \item<4-> Ramas para experimentar
                    \item<5-> Trazabilidad total
                \end{itemize}
            \end{darkbox}
        \end{column}
    \end{columns}
    
    \vspace{0.8cm}
    \pause
    \centering
    {\large\color{gludOrange} Hoy configuraremos todo lo necesario para trabajar profesionalmente}
\end{frame}

\quoteslide{The best time to start using version control was yesterday. The second best time is now.}{Desarrollador Anónimo}

% SECTION 2: INSTALACIÓN DE GIT

\section{Instalación de Git}

\begin{frame}{Instalación en Linux}
    \vspace{0.3cm}
    
    \begin{columns}[T]
        \begin{column}{0.48\textwidth}
            \begin{codebox}[Debian/Ubuntu]
                \ttfamily\small\color{gludWhite}
                sudo apt update\\
                sudo apt install git -y
            \end{codebox}
            
            \vspace{0.4cm}
            
            \begin{codebox}[Fedora/RHEL]
                \ttfamily\small\color{gludWhite}
                sudo dnf install git -y
            \end{codebox}
        \end{column}
        \pause
        \begin{column}{0.48\textwidth}
            \begin{codebox}[Arch Linux]
                \ttfamily\small\color{gludWhite}
                sudo pacman -Sy git
            \end{codebox}
            
            \vspace{0.4cm}
            
            \begin{alertbox}[Verificación]
                Después de instalar:
                \begin{center}
                    \ttfamily\color{gludDark}
                    git -{}-version
                \end{center}
            \end{alertbox}
        \end{column}
    \end{columns}  
\end{frame}

\begin{frame}{Configuración Inicial de Git}
    
    \step{1}{Configura tu identidad (obligatorio)}
    \vspace{-0.2cm}
    \begin{codebox}
        \ttfamily\color{gludWhite}
        git config -{}-global user.name "Tu Nombre"\\
        git config -{}-global user.email "tu.email@ejemplo.com"
    \end{codebox}
    
    \pause
    
    \step{2}{Configura la rama por defecto}
    \vspace{-0.2cm}
    \begin{codebox}
        \ttfamily\color{gludWhite}
        git config -{}-global init.defaultBranch main
    \end{codebox}
    
    \pause
    
    \step{3}{Configura tu editor preferido}
    \vspace{-0.2cm}
    \begin{codebox}
        \ttfamily\color{gludWhite}
        git config -{}-global core.editor nano\\
        \# Alternativas: vim, emacs, code -{}-wait
    \end{codebox}
\end{frame}

\begin{frame}{Verificar Configuración}
    \vspace{0.3cm}
    
    \begin{codebox}[Ver toda la configuración]
        \ttfamily\color{gludWhite}
        git config -{}-list -{}-global
    \end{codebox}
    
    \vspace{0.5cm}
    \pause
    
    \begin{codebox}[Salida esperada]
        \ttfamily\small\color{gludWhite}
        user.name=Tu Nombre\\
        user.email=tu.email@ejemplo.com\\
        init.defaultbranch=main\\
        core.editor=nano
    \end{codebox}
\end{frame}

% SECTION 3: INSTALACIÓN JDK Y MAVEN

\section{Java y Maven}

\begin{frame}{Instalación de JDK 21 y Maven}
    \vspace{0.3cm}
    
    \begin{codebox}[Debian/Ubuntu]
        \ttfamily\small\color{gludWhite}
        sudo apt update\\
        sudo apt install openjdk-21-jdk maven -y
    \end{codebox}
    
    \pause
    
    \begin{codebox}[Fedora/RHEL]
        \ttfamily\small\color{gludWhite}
        sudo dnf install java-21-openjdk-devel maven -y
    \end{codebox}
    
    \pause
    
    \begin{codebox}[Verificación]
        \ttfamily\small\color{gludWhite}
        java -version\\
        mvn -v
    \end{codebox}
    
    \vspace{0.3cm}
    \pause
    
    \centering
    {\color{gludLightGray} Deberías ver Java 21+ y Maven 3.X}
\end{frame}

\begin{frame}{Variables de Entorno}
    \vspace{0.3cm}
    
    \begin{orangebox}[¿Qué son?]
        Las \highlight{variables de entorno} son valores que el sistema operativo utiliza para configurar el comportamiento de programas y scripts.
    \end{orangebox}
    
    \vspace{0.4cm}
    \pause
    
    \begin{columns}[T]
        \begin{column}{0.48\textwidth}
            \textbf{JAVA\_HOME}
            
            \vspace{0.2cm}
            
            Apunta a la instalación de Java
            
            \begin{codebox}
                \ttfamily\tiny\color{gludWhite}
                /usr/lib/jvm/\\
                \hspace{0.5cm}java-21-openjdk
            \end{codebox}
        \end{column}
        \pause
        \begin{column}{0.48\textwidth}
            \textbf{PATH}
            
            \vspace{0.2cm}
            
            Directorios donde buscar ejecutables
            
            \begin{codebox}
                \ttfamily\tiny\color{gludWhite}
                \$JAVA\_HOME/bin:\\
                /usr/local/bin:\\
                /usr/bin
            \end{codebox}
        \end{column}
    \end{columns}
\end{frame}

\begin{frame}{Configurar Variables de Entorno}
    
    \step{1}{Edita tu archivo de configuración de shell}
    \vspace{-0.2cm}
    \begin{codebox}
        \ttfamily\color{gludWhite}
        nano \textasciitilde/.bashrc \hspace{1cm}\# o \textasciitilde/.zshrc
    \end{codebox}
    
    \pause
    
    \step{2}{Agrega al final del archivo}
    \vspace{-0.2cm}
    \begin{codebox}
        \ttfamily\small\color{gludWhite}
        export JAVA\_HOME=/usr/lib/jvm/java-21-openjdk\\
        export PATH="\$JAVA\_HOME/bin:\$PATH"
    \end{codebox}
    
    \pause
    
    \step{3}{Recarga la configuración}
    \vspace{-0.2cm}
    \begin{codebox}
        \ttfamily\color{gludWhite}
        source \textasciitilde/.bashrc
    \end{codebox}
\end{frame}

% SECTION 4: ARQUITECTURA DE GIT

\section{Arquitectura de Git}

\begin{frame}{Las Cuatro Áreas de Git}
    \vspace{0.3cm}
    \centering
    
    \begin{tikzpicture}[scale=0.85]
        % Working Directory
        \fill[gludPeach] (0,0) rectangle (3,2);
        \node[text=gludDark,font=\bfseries\small] at (1.5,1.5) {Working};
        \node[text=gludDark,font=\bfseries\small] at (1.5,1) {Directory};
        \node[text=gludGray,font=\tiny] at (1.5,0.3) {Archivos editables};
        
        \pause
        
        % Arrow 1
        \draw[->,thick,gludOrange] (3.2,1) -- (4,1);
        \node[text=gludWhite,font=\tiny] at (3.6,1.4) {git add};
        
        % Staging Area
        \fill[gludOrange] (4.2,0) rectangle (7.2,2);
        \node[text=gludWhite,font=\bfseries\small] at (5.7,1.5) {Staging};
        \node[text=gludWhite,font=\bfseries\small] at (5.7,1) {Area};
        \node[text=gludPeach,font=\tiny] at (5.7,0.3) {Preparados};
        
        \pause
        
        % Arrow 2
        \draw[->,thick,gludOrange] (7.4,1) -- (8.2,1);
        \node[text=gludWhite,font=\tiny] at (7.8,1.4) {git commit};
        
        % Local Repository
        \fill[gludDarker] (8.4,0) rectangle (11.4,2);
        \draw[gludOrange,thick] (8.4,0) rectangle (11.4,2);
        \node[text=gludWhite,font=\bfseries\small] at (9.9,1.5) {Local};
        \node[text=gludWhite,font=\bfseries\small] at (9.9,1) {Repository};
        \node[text=gludLightGray,font=\tiny] at (9.9,0.3) {Historial local};
        
        \pause
        
        % Arrow 3
        \draw[->,thick,gludOrange] (11.6,1) -- (12.4,1);
        \node[text=gludWhite,font=\tiny] at (12,1.4) {git push};
        
        % Remote Repository
        \fill[gludOrangeDark] (12.6,0) rectangle (15.6,2);
        \node[text=gludWhite,font=\bfseries\small] at (14.1,1.5) {Remote};
        \node[text=gludWhite,font=\bfseries\small] at (14.1,1) {Repository};
        \node[text=gludPeach,font=\tiny] at (14.1,0.3) {GitHub/GitLab};
    \end{tikzpicture}
    
    \vspace{0.6cm}
    
    \begin{columns}[T]
        \begin{column}{0.23\textwidth}
            \centering
            \circled{1} \highlight{Working}
            
            \vspace{0.1cm}
            {\tiny Modificas archivos}
        \end{column}
        \begin{column}{0.23\textwidth}
            \centering
            \circled{2} \highlight{Staging}
            
            \vspace{0.1cm}
            {\tiny Seleccionas cambios}
        \end{column}
        \begin{column}{0.23\textwidth}
            \centering
            \circled{3} \highlight{Local Repo}
            
            \vspace{0.1cm}
            {\tiny Guardas commits}
        \end{column}
        \begin{column}{0.23\textwidth}
            \centering
            \circled{4} \highlight{Remote}
            
            \vspace{0.1cm}
            {\tiny Sincronizas}
        \end{column}
    \end{columns}
\end{frame}

\begin{frame}{Working Directory}
    \vspace{0.3cm}
    
    \begin{orangebox}[Definición]
        El \highlight{Working Directory} es tu carpeta de trabajo donde modificas archivos libremente.
    \end{orangebox}
    
    \vspace{0.4cm}
    \pause
    
    \begin{columns}[T]
        \begin{column}{0.48\textwidth}
            \textbf{Características:}
            \begin{itemize}
                \item Archivos sin rastrear
                \item Archivos modificados
                \item Puedes editar libremente
            \end{itemize}
        \end{column}
        \pause
        \begin{column}{0.48\textwidth}
            \begin{codebox}[Ver estado]
                \ttfamily\small\color{gludWhite}
                git status
            \end{codebox}
            
            \vspace{0.3cm}
            
            {\small\color{gludLightGray} Muestra archivos modificados y sin rastrear}
        \end{column}
    \end{columns}
\end{frame}

\begin{frame}{Staging Area (Index)}
    \vspace{0.1cm}
    
    \begin{orangebox}[Definición]
        El \highlight{Staging Area} es donde preparas exactamente qué cambios quieres guardar en el próximo commit.
    \end{orangebox}
    
    \vspace{0.2cm}
    \pause
    
    \begin{columns}[T]
        \begin{column}{0.48\textwidth}
            \begin{codebox}[Agregar archivos]
                \ttfamily\small\color{gludWhite}
                git add archivo.txt\\[4pt]
                git add src/\\[4pt]
                git add .
            \end{codebox}
        \end{column}
        \pause
        \begin{column}{0.48\textwidth}
            \begin{codebox}[Agregar selectivamente]
                \ttfamily\small\color{gludWhite}
                git add -p archivo.txt
            \end{codebox}
            
            \vspace{0.3cm}
            
            {\small\color{gludLightGray} Te permite elegir qué partes agregar}
        \end{column}
    \end{columns}
    
    \vspace{0.5cm}
    \pause
    
    \centering
    {\color{gludOrangeLight} El staging permite commits \textbf{atómicos} y bien organizados}
\end{frame}

\begin{frame}{Comparando Cambios}
    
    \step{1}{Ver cambios en Working Directory}
    \vspace{-0.2cm}
    \begin{codebox}
        \ttfamily\color{gludWhite}
        git diff
    \end{codebox}
    
    \pause
    
    \step{2}{Ver cambios en Staging}
    \vspace{-0.2cm}
    \begin{codebox}
        \ttfamily\color{gludWhite}
        git diff -{}-staged
    \end{codebox}
    
    \pause
    
    \step{3}{Ver cambios entre commits}
    \vspace{-0.2cm}
    \begin{codebox}
        \ttfamily\color{gludWhite}
        git diff HEAD\textasciitilde{}1 HEAD
    \end{codebox}
    
    \pause
    
    \centering
    {\color{gludLightGray} \texttt{git diff} es tu mejor amigo para revisar cambios}
\end{frame}

\begin{frame}{Local Repository}
    \vspace{0.1cm}
    
    \begin{orangebox}[Definición]
        El \highlight{Local Repository} almacena el historial completo de commits en tu máquina.
    \end{orangebox}

    \pause
    
    \begin{columns}[T]
        \begin{column}{0.48\textwidth}
            \begin{codebox}[Crear commit]
                \ttfamily\small\color{gludWhite}
                git commit -m "mensaje"
            \end{codebox}
            
            \vspace{0.1cm}
            
            \begin{codebox}[Ver historial]
                \ttfamily\small\color{gludWhite}
                git log -{}-oneline\\
                git log -{}-graph -{}-all
            \end{codebox}
        \end{column}
        \pause
        \begin{column}{0.48\textwidth}
            \begin{codebox}[Ver un commit]
                \ttfamily\small\color{gludWhite}
                git show <hash>
            \end{codebox}
            
            \vspace{0.1cm}
            
            \begin{alertbox}[Tip]
                Escribe mensajes descriptivos y claros
            \end{alertbox}
        \end{column}
    \end{columns}
\end{frame}

\begin{frame}{Remote Repository}
    \vspace{-0.2cm}
    \begin{orangebox}[Definición]
        Es una copia de tu repositorio alojada en un servidor (GitHub, GitLab, etc.).
    \end{orangebox}
    
    \pause
    
    \begin{codebox}[Conectar con remoto]
        \ttfamily\small\color{gludWhite}
        git remote add origin https://github.com/usuario/repo.git
    \end{codebox}
    
    \pause
    
    \begin{codebox}[Enviar cambios]
        \ttfamily\small\color{gludWhite}
        git push -u origin main
    \end{codebox}
    
    \pause
    
    \begin{codebox}[Clonar repositorio]
        \ttfamily\small\color{gludWhite}
        git clone https://github.com/usuario/repo.git
    \end{codebox}
\end{frame}

% SECTION 5: PRÁCTICA GUIADA

\section{Práctica Guiada}

\begin{frame}{Creando tu Primer Repositorio}
    \vspace{0.3cm}
    
    \step{1}{Crear directorio y entrar}
    
    \begin{codebox}
        \ttfamily\color{gludWhite}
        mkdir mi-primer-repo \&\& cd mi-primer-repo
    \end{codebox}
    
    \pause
    
    \step{2}{Inicializar Git}
    
    \begin{codebox}
        \ttfamily\color{gludWhite}
        git init
    \end{codebox}
    
    \pause
    
    \begin{alertbox}[Resultado]
        Se crea una carpeta oculta \texttt{.git/} que contiene toda la información del repositorio.
    \end{alertbox}
\end{frame}

\begin{frame}{Primer Commit}

    \step{3}{Crear archivo README}
    
    \begin{codebox}
        \ttfamily\color{gludWhite}
        echo "\# Mi Primer Repositorio" > README.md
    \end{codebox}
    
    \pause
    
    \step{4}{Agregar al staging}
    
    \begin{codebox}
        \ttfamily\color{gludWhite}
        git add README.md
    \end{codebox}
    
    \pause
    
    \step{5}{Crear commit}
    
    \begin{codebox}
        \ttfamily\color{gludWhite}
        git commit -m "feat: agregar README inicial"
    \end{codebox}
\end{frame}

\begin{frame}{Gitignore: Ignorar Archivos}
    
    \begin{orangebox}[¿Qué es .gitignore?]
        Archivo que especifica qué archivos NO deben ser rastreados por Git.
    \end{orangebox}

    \pause
    
    \begin{codebox}[Crear .gitignore para Java/Maven]
        \ttfamily\small\color{gludWhite}
        target/\\
        *.class\\
        .idea/\\
        .vscode/\\
        *.log
    \end{codebox}

    \centering
    {\color{gludLightGray} Siempre crea .gitignore al inicio del proyecto}
\end{frame}


\begin{frame}{Flujo Completo de Trabajo}
    
    \begin{codebox}[Secuencia típica]
        \ttfamily\small\color{gludWhite}
        \# 1. Hacer cambios en archivos\\
        vim archivo.java\\[6pt]
        \# 2. Ver qué cambió\\
        git status\\
        git diff\\[6pt]
        \# 3. Agregar al staging\\
        git add archivo.java\\[6pt]
        \# 4. Crear commit\\
        git commit -m "feat: implementar nueva funcionalidad"\\[6pt]
        \# 5. Enviar al remoto\\
        git push
    \end{codebox}
\end{frame}

\begin{frame}{Comandos de Ayuda}
    \vspace{0.3cm}
    
    \begin{codebox}[Estructura del ejercicio]
        \ttfamily\small\color{gludWhite}
        mkdir proyecto-java \&\& cd proyecto-java\\
        git init\\
        mkdir -p src/main/java/com/ejemplo\\
        touch README.md .gitignore\\
        git add README.md\\
        git commit -m "docs: agregar README"\\
        git add .gitignore\\
        git commit -m "chore: configurar gitignore"\\
        \# ... continúa con más commits
    \end{codebox}
\end{frame}

% SECTION 8: RESUMEN

\section{Resumen}

\begin{frame}{Lo que Aprendimos}
    \vspace{0.3cm}
    
    \begin{columns}[T]
        \begin{column}{0.48\textwidth}
            \textbf{Instalación:}
            \begin{itemize}
                \item Git en Linux
                \item JDK 17 + Maven
                \item Variables de entorno
                \item Configuración global
            \end{itemize}
            
            \vspace{0.3cm}
            
            \textbf{Conceptos:}
            \begin{itemize}
                \item Working Directory
                \item Staging Area
                \item Local Repository
                \item Remote Repository
            \end{itemize}
        \end{column}
        \begin{column}{0.48\textwidth}
            \textbf{Comandos:}
            \begin{itemize}
                \item \inlinecode{git init}
                \item \inlinecode{git add / git add -p}
                \item \inlinecode{git commit}
                \item \inlinecode{git diff / -{}-staged}
                \item \inlinecode{git push / pull / fetch}
                \item \inlinecode{git remote add}
                \item \inlinecode{git clone}
            \end{itemize}
        \end{column}
    \end{columns}
\end{frame}

\begin{frame}{Recursos Útiles}
    \vspace{0.5cm}
    
    \begin{columns}[T]
        \begin{column}{0.48\textwidth}
            \textbf{Documentación:}
            \begin{itemize}
                \item Pro Git (libro oficial)
                \item \url{https://git-scm.com/docs}
                \item \url{https://training.github.com}
            \end{itemize}
        \end{column}
        \begin{column}{0.48\textwidth}
            \textbf{Práctica:}
            \begin{itemize}
                \item \url{https://learngitbranching.js.org}
                \item \url{https://gitexercises.fracz.com}
                \item GitHub Student Developer Pack
            \end{itemize}
        \end{column}
    \end{columns}
    
    \vspace{1cm}
    
    \centering
    {\large\color{gludOrange} La práctica hace al maestro}
\end{frame}

% Next class preview

\begin{frame}[plain]
\setbeamercolor{background canvas}{bg=gludPeach}
\vspace{1.5cm}
\centering

{\Huge\color{gludOrange}\bfseries Próxima Clase}

\vspace{0.5cm}

{\Large\color{gludDark} Ramas y Desarrollo Colaborativo}

\vspace{1cm}

{\large\color{gludWhite} Estados del archivo (Untracked, Staged, Committed). Conventional Commits. Estructura de proyecto del stack elegido. Configuración de .gitignore apropiado (gitignore.io).}
\end{frame}

% Despedida

\thankslide{¡Gracias!\\[0.3cm]{\hspace*{1cm}\Large Nos vemos en la próxima clase}}

\end{document}
