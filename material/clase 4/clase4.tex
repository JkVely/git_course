\documentclass[aspectratio=169,10pt]{beamer}
\usepackage{../theme}
\usepackage{colortbl}

% METADATA

\title{Trabajando con Ramas}
\subtitle{Branches, Git Flow y Estrategias de Desarrollo}
\author{GLUD --- Grupo GNU/Linux Universidad Distrital}
\date{Control de Versiones y Desarrollo Colaborativo}

% DOCUMENT

\begin{document}

% Portada
\titleframe[Clase 4 · Trabajando con Ramas]

\contentsframe

% SECTION 1: INTRODUCCIÓN A RAMAS

\section{¿Qué son las Ramas?}

\begin{frame}{El Concepto de Ramas}
    \vspace{0.3cm}
    
    \begin{orangebox}[Definición]
        Una \highlight{rama} (branch) es una línea independiente de desarrollo que permite trabajar en características sin afectar el código principal.
    \end{orangebox}
    
    \vspace{0.4cm}
    \pause
    
    \begin{columns}[T]
        \begin{column}{0.48\textwidth}
            \textbf{¿Por qué usar ramas?}
            \begin{itemize}
                \item Aislar nuevas características
                \item Experimentar sin riesgos
                \item Trabajo paralelo en equipo
                \item Mantener código estable
            \end{itemize}
        \end{column}
        \pause
        \begin{column}{0.48\textwidth}
            \textbf{Escenarios comunes:}
            \begin{itemize}
                \item Desarrollar nueva funcionalidad
                \item Corregir bugs urgentes
                \item Probar ideas experimentales
                \item Preparar releases
            \end{itemize}
        \end{column}
    \end{columns}
\end{frame}

\begin{frame}{Visualizando Ramas}
    
    \begin{center}
        \begin{tikzpicture}[
            commit/.style={circle, fill=gludOrange, text=white, font=\tiny\bfseries, minimum size=0.6cm},
            branch/.style={rectangle, rounded corners, fill=gludDarker, text=gludOrange, font=\tiny\ttfamily, inner sep=3pt},
            arrow/.style={->, thick, gludOrangeLight}
        ]
            % Main branch
            \node[commit] (c1) at (0,0) {C1};
            \node[commit] (c2) at (2,0) {C2};
            \node[commit] (c3) at (4,0) {C3};
            \node[commit] (c6) at (6,0) {C6};
            
            % Feature branch
            \node[commit] (c4) at (4,2) {C4};
            \node[commit] (c5) at (6,2) {C5};
            
            % Connections
            \draw[arrow] (c1) -- (c2);
            \draw[arrow] (c2) -- (c3);
            \draw[arrow] (c3) -- (c6);
            \draw[arrow] (c3) -- (c4);
            \draw[arrow] (c4) -- (c5);
            
            % Branch labels
            \node[branch] at (0,-0.8) {main};
            \node[branch] at (6,2.8) {feature};
            
            % Annotations
            \node[text=gludWhite, font=\scriptsize] at (2,-1.5) {Rama principal};
            \node[text=gludWhite, font=\scriptsize] at (5,3.5) {Desarrollo aislado};
        \end{tikzpicture}
    \end{center}
    
    \centering
    {\color{gludOrangeLight} Las ramas permiten desarrollo paralelo sin interferencias}
\end{frame}

\quoteslide{A branch in Git is simply a lightweight movable pointer to a commit.}{Pro Git Book}

% SECTION 2: COMANDOS BÁSICOS DE RAMAS

\section{Comandos Básicos}

\begin{frame}{Ver Ramas Existentes}
    
    \begin{orangebox}[git branch]
        Lista todas las ramas locales. La rama actual se marca con \texttt{*}
    \end{orangebox}
    
    \pause
    
    \begin{codebox}[Listar ramas locales]
        \ttfamily\color{gludWhite}
        git branch
    \end{codebox}
    
    \pause
    
    \begin{codebox}[Salida]
        \ttfamily\color{gludWhite}
        \ \ develop\\
        * main\\
        \ \ feature/login
    \end{codebox}
\end{frame}

\begin{frame}{Ver Ramas con Detalles}
    \begin{columns}[T]
        \begin{column}{0.48\textwidth}
            \begin{codebox}[Ver todas las ramas]
                \ttfamily\small\color{gludWhite}
                git branch -{-}all
            \end{codebox}
        \end{column}
        \begin{column}{0.48\textwidth}
            \begin{codebox}[Ver con último commit]
                \ttfamily\small\color{gludWhite}
                git branch -v
            \end{codebox}
        \end{column}
    \end{columns}
\end{frame}

\begin{frame}{Crear Ramas}
    
    \step{1}{Crear rama sin cambiar a ella}
    
    \begin{codebox}
        \ttfamily\color{gludWhite}
        git branch nombre-rama
    \end{codebox}
    
    \pause
    
    \step{2}{Crear y cambiar a la rama (método clásico)}
    
    \begin{codebox}
        \ttfamily\color{gludWhite}
        git checkout -b nombre-rama
    \end{codebox}
    
    \pause
    
    \step{3}{Crear y cambiar a la rama (método moderno)}
    
    \begin{codebox}
        \ttfamily\color{gludWhite}
        git switch -c nombre-rama
    \end{codebox}
    
    \vspace{0.3cm}
\end{frame}

\begin{frame}{Cambiar entre Ramas}

    \begin{columns}[T]
        \begin{column}{0.48\textwidth}
            \begin{codebox}[Método clásico]
                \ttfamily\small\color{gludWhite}
                git checkout nombre-rama
            \end{codebox}
            
            {\small\color{gludLightGray} Comando tradicional}
        \end{column}
        \pause
        \begin{column}{0.48\textwidth}
            \begin{codebox}[Método moderno]
                \ttfamily\small\color{gludWhite}
                git switch nombre-rama
            \end{codebox}
            
            {\small\color{gludLightGray} Más intuitivo y seguro}
        \end{column}
    \end{columns}
    \pause
    
    \begin{codebox}[Volver a la rama anterior]
        \ttfamily\color{gludWhite}
        git switch -
    \end{codebox}

    \pause
    
    \begin{orangebox}[Importante]
        {\color{black}Antes de cambiar de rama, asegúrate de que tu \highlight{working directory} esté limpio o haz commit de tus cambios.}
    \end{orangebox}
\end{frame}

\begin{frame}{Eliminar Ramas}
    
    \step{1}{Eliminar rama fusionada (seguro)}
    
    \begin{codebox}
        \ttfamily\color{gludWhite}
        git branch -d nombre-rama {\small\color{gludLightGray} \#Git verifica que la rama esté fusionada}
    \end{codebox}
    
    \pause
    
    \step{2}{Eliminar rama forzosamente}
    
    \begin{codebox}
        \ttfamily\color{gludWhite}
        git branch -D nombre-rama {\small\color{gludLightGray} \#Elimina aunque no esté fusionada. \textbf{¡Cuidado!}}
    \end{codebox}
    
    \pause
    
    \begin{alertbox}[Advertencia]
        {\color{black}No puedes eliminar la rama en la que estás actualmente. Cambia a otra rama primero.}
    \end{alertbox}
\end{frame}

\begin{frame}{Renombrar Ramas}

    \begin{codebox}[Renombrar rama actual]
        \ttfamily\color{gludWhite}
        git branch -m nuevo-nombre
    \end{codebox}

    \pause
    
    \begin{codebox}[Renombrar otra rama]
        \ttfamily\color{gludWhite}
        git branch -m nombre-viejo nombre-nuevo
    \end{codebox}

    \pause
    
    \begin{orangebox}[Ejemplo práctico]
        {\color{black}
        \begin{itemize}
            \item {\color{black}\texttt{git branch -m master main} --- Actualizar nomenclatura
            \item \texttt{git branch -m feat/login feature/login} --- Corregir formato}
        \end{itemize}
        }
    \end{orangebox}
\end{frame}

\begin{frame}[fragile]{Ejemplo Práctico: Flujo Básico}
    \vspace{0.3cm}
    
    \begin{codebox}[Crear y trabajar en una rama]
        \ttfamily\small\color{gludWhite}
        \# Crear nueva rama para característica\\
        git switch -c feature/calculadora\\
        \\
        \# Hacer cambios y commits\\
        git add calculadora.py\\
        git commit -m "feat: implementar suma"\\
        \\
        \# Volver a main\\
        git switch main\\
        \\
        \# Eliminar rama (después de fusionar)\\
        git branch -d feature/calculadora
    \end{codebox}
\end{frame}

% SECTION 3: ESTRATEGIAS DE RAMAS

\section{Estrategias de Ramas}

\begin{frame}{¿Por qué Necesitamos Estrategias?}
    \vspace{0.3cm}
    
    \begin{orangebox}[El Problema]
        {\color{black}Sin una estrategia clara, los equipos crean ramas sin control, generando caos y conflictos.}
    \end{orangebox}
    
    \vspace{0.4cm}
    \pause
    
    \textbf{Consecuencias de no tener estrategia:}
    \begin{itemize}
        \item Nombres inconsistentes de ramas
        \item No se sabe qué está en producción
        \item Conflictos difíciles de resolver
        \item Código sin revisar en producción
        \item Dificultad para rastrear cambios
    \end{itemize}
    
    \vspace{0.4cm}
    \pause
    
    \centering
    {\color{gludOrangeLight} \textbf{Solución:} Adoptar una estrategia de ramas estándar}
\end{frame}

\begin{frame}{Estrategias Principales}
    \vspace{0.3cm}
    
    \begin{columns}[T]
        \begin{column}{0.48\textwidth}
            \begin{darkbox}[Git Flow]
                {\color{gludWhite}
                \textbf{Creado por:} Vincent Driessen (2010)
                
                \vspace{0.2cm}
                
                \textbf{Características:}
                \begin{itemize}
                    \item Múltiples ramas permanentes
                    \item Estructura muy definida
                    \item Ideal para releases planificados
                    \item Más complejo pero robusto
                \end{itemize}
                }
            \end{darkbox}
        \end{column}
        \pause
        \begin{column}{0.48\textwidth}
            \begin{darkbox}[Feature Branch]
                {\color{gludWhite}
                \textbf{También llamado:} GitHub Flow
                
                \vspace{0.2cm}
                
                \textbf{Características:}
                \begin{itemize}
                    \item Una rama principal (main)
                    \item Ramas por característica
                    \item Más simple y flexible
                    \item Más propensa a errores
                \end{itemize}
                }
            \end{darkbox}
        \end{column}
    \end{columns}
    
    \vspace{0.5cm}
    
    \centering
    {\color{gludOrangeLight} Ambas son válidas --- la elección depende del proyecto}
\end{frame}

% SECTION 4: GIT FLOW

\section{Git Flow}

\begin{frame}{Git Flow: Estructura}
    \vspace{0.3cm}
    
    \begin{orangebox}[Ramas Permanentes]
        \begin{itemize}
            {\color{black}
            \item \texttt{main} --- Código en producción, siempre estable
            \item \texttt{develop} --- Integración de características para próximo release
            }
        \end{itemize}
    \end{orangebox}
    
    \vspace{0.3cm}
    \pause
    
    \begin{orangebox}[Ramas Temporales]
        \begin{itemize}
            {\color{black}
            \item \texttt{feature/*} --- Nuevas características (desde develop)
            \item \texttt{release/*} --- Preparación de release (desde develop)
            \item \texttt{hotfix/*} --- Correcciones urgentes (desde main)
            }
        \end{itemize}
    \end{orangebox}
\end{frame}

\begin{frame}{Git Flow: Diagrama Visual}
    
    \begin{center}
        \begin{figure}[H]
            \includegraphics[width=0.54\textwidth]{img/git-flow-diagram.png}
            \caption{\scriptsize Fuente: \url{https://www.atlassian.com/es/git/tutorials/comparing-workflows/gitflow-workflow}}
        \end{figure}
    \end{center}
\end{frame}

\begin{frame}{Git Flow: Ciclo de Vida Feature}
    
    \step{1}{Crear feature desde develop}
    
    \begin{codebox}
        \ttfamily\small\color{gludWhite}
        git switch develop\\
        git switch -c feature/nueva-funcionalidad
    \end{codebox}
    
    \pause
    
    \step{2}{Desarrollar y hacer commits}
    
    \pause
    
    \step{3}{Finalizar feature (fusionar a develop)}
    
    \begin{codebox}
        \ttfamily\small\color{gludWhite}
        git switch develop\\
        git merge feature/nueva-funcionalidad\\
        git branch -d feature/nueva-funcionalidad
    \end{codebox}
\end{frame}

\begin{frame}{Git Flow: Ciclo de Vida Hotfix}
    \step{1}{Crear hotfix desde main}
    
    \begin{codebox}
        \ttfamily\small\color{gludWhite}
        git switch main \& git switch -c hotfix/corregir-bug-critico
    \end{codebox}
    
    \pause
    
    \step{2}{Corregir y commit}
    
    \pause
    
    \step{3}{Fusionar a main y develop}
    
    \begin{codebox}
        \ttfamily\small\color{gludWhite}
        git switch main\\
        git merge hotfix/corregir-bug-critico\\
        git switch develop\\
        git merge hotfix/corregir-bug-critico\\
        git branch -d hotfix/corregir-bug-critico
    \end{codebox}
\end{frame}

\begin{frame}{Git Flow: El Addon git-flow-avh}
    \begin{orangebox}[git-flow (AVH Edition)]
        {\color{black}Extensión oficial que simplifica los comandos de Git Flow creada por \highlight{Peter van der Does}.}
    \end{orangebox}
    
    \pause
    
    \begin{codebox}[Instalación]
        \ttfamily\small\color{gludWhite}
        \# Linux (Debian/Ubuntu)\\
        sudo apt install git-flow\\
        \\
        \# macOS\\
        brew install git-flow-avh\\
        \\
        \# Windows (con Chocolatey)\\
        choco install git-flow-avh
    \end{codebox}
\end{frame}

\begin{frame}{git-flow: Inicialización}
    \vspace{0.3cm}
    
    \begin{codebox}[Inicializar Git Flow en repositorio]
        \ttfamily\color{gludWhite}
        git flow init
    \end{codebox}
    
    \vspace{0.3cm}
    \pause
    
    \begin{codebox}[Preguntas interactivas]
        \ttfamily\tiny\color{gludWhite}
        Which branch should be used for bringing forth production releases?\\
        \ \ \ Branch name for production releases: [main]\\
        \\
        Which branch should be used for integration of the "next release"?\\
        \ \ \ Branch name for "next release" development: [develop]\\
        \\
        How to name your supporting branch prefixes?\\
        \ \ \ Feature branches? [feature/]\\
        \ \ \ Release branches? [release/]\\
        \ \ \ Hotfix branches? [hotfix/]
    \end{codebox}
    
    \vspace{0.3cm}
    
    {\scriptsize\color{gludLightGray} Acepta los valores por defecto presionando Enter}
\end{frame}

\begin{frame}{git-flow: Comandos Simplificados}
    \vspace{0.3cm}
    
    \begin{columns}[T]
        \begin{column}{0.48\textwidth}
            \begin{darkbox}[Feature]
                {\color{gludWhite}
                \begin{codebox}
                    \ttfamily\tiny\color{gludWhite}
                    \# Iniciar\\
                    git flow feature start login\\
                    \\
                    \# Finalizar\\
                    git flow feature finish login
                \end{codebox}
                }
            \end{darkbox}
        \end{column}
        \pause
        \begin{column}{0.48\textwidth}
            \begin{darkbox}[Hotfix]
                {\color{gludWhite}
                \begin{codebox}
                    \ttfamily\tiny\color{gludWhite}
                    \# Iniciar\\
                    git flow hotfix start 1.0.1\\
                    \\
                    \# Finalizar\\
                    git flow hotfix finish 1.0.1
                \end{codebox}
                }
            \end{darkbox}
        \end{column}
    \end{columns}
    
    \vspace{0.4cm}
    \pause
    
    \begin{codebox}[Release]
        \ttfamily\small\color{gludWhite}
        git flow release start 2.0.0\\
        git flow release finish 2.0.0
    \end{codebox}
    
    \vspace{0.3cm}
    
    \centering
    {\color{gludOrangeLight} \textbf{finish} automáticamente fusiona y elimina la rama}
\end{frame}

\begin{frame}{git-flow: Comparación de Comandos}
    \begin{columns}[T]
        \begin{column}{0.48\textwidth}
            \begin{darkbox}[Sin git-flow]
                {\color{gludWhite}
                \begin{codebox}
                    \ttfamily\tiny\color{gludWhite}
                    git switch develop\\
                    git switch -c feature/login\\
                    \# ... trabajo ...\\
                    git switch develop\\
                    git merge feature/login\\
                    git branch -d feature/login
                \end{codebox}
                
                {\scriptsize 5 comandos}
                }
            \end{darkbox}
        \end{column}
        \begin{column}{0.48\textwidth}
            \begin{darkbox}[Con git-flow]
                {\color{gludWhite}
                \begin{codebox}
                    \ttfamily\tiny\color{gludWhite}
                    git flow feature start login\\
                    \# ... trabajo ...\\
                    git flow feature finish login
                \end{codebox}
                
                {\scriptsize 2 comandos}
                }
            \end{darkbox}
        \end{column}
    \end{columns}
    
    \vspace{0.5cm}
    \pause
    
    \begin{orangebox}[Ventaja]
        {\color{black}Menos errores, más eficiencia, nombres consistentes automáticamente.}
    \end{orangebox}
\end{frame}

\begin{frame}{Git Flow: Ventajas y Desventajas}
    \vspace{0.3cm}
    
    \begin{columns}[T]
        \begin{column}{0.48\textwidth}
            \textbf{\color{gludOrange} Ventajas}
            \begin{itemize}
                \item Muy estructurado y organizado
                \item Producción siempre estable (main)
                \item Ideal para releases planificados
                \item Hotfixes bien definidos
                \item Menos conflictos
            \end{itemize}
        \end{column}
        \pause
        \begin{column}{0.48\textwidth}
            \textbf{\color{gludOrange} Desventajas}
            \begin{itemize}
                \item Más complejo de aprender
                \item Requiere disciplina del equipo
                \item No ideal para CI/CD continuo
                \item Overhead para proyectos pequeños
                \item Dos ramas permanentes
            \end{itemize}
        \end{column}
    \end{columns}
    
    \vspace{0.5cm}
    \pause
    
    \centering
    {\color{gludOrangeLight} \textbf{Ideal para:} Proyectos grandes, equipos medianos/grandes, releases programados}
\end{frame}

% SECTION 5: FEATURE BRANCH WORKFLOW

\section{Feature Branch Workflow}

\begin{frame}{Feature Branch: Estructura}
    \vspace{0.3cm}
    
    \begin{orangebox}[Concepto]
        {\color{black}Toda nueva característica o corrección se desarrolla en una rama separada desde \texttt{main}.}
    \end{orangebox}
    
    \vspace{0.4cm}
    \pause
    
    \textbf{Reglas básicas:}
    \begin{itemize}
        \item Una sola rama permanente: \texttt{main}
        \item Cada developer crea ramas para sus tareas
        \item Las ramas se fusionan a \texttt{main} cuando están listas
        \item \texttt{main} siempre debe estar desplegable
    \end{itemize}
    
    \vspace{0.4cm}
    \pause
    
    \centering
    {\color{gludOrangeLight} También conocido como \textbf{GitHub Flow}}
\end{frame}

\begin{frame}{Feature Branch: Diagrama Visual}
    \vspace{0.3cm}
    
    \begin{center}
        \begin{tikzpicture}[
            commit/.style={circle, fill=gludOrange, text=white, font=\tiny\bfseries, minimum size=0.5cm},
            branch/.style={rectangle, rounded corners, fill=gludDarker, text=gludOrange, font=\tiny\ttfamily, inner sep=2pt},
            arrow/.style={->, thick}
        ]
            % Main branch
            \node[commit] (m1) at (0,0) {C1};
            \node[commit] (m2) at (2,0) {C2};
            \node[commit] (m3) at (4,0) {C3};
            \node[commit] (m4) at (6,0) {C4};
            \node[commit] (m5) at (8,0) {C5};
            
            % Developer 1 branch
            \node[commit] (d1a) at (2,1.5) {F1};
            \node[commit] (d1b) at (3,1.5) {F2};
            
            % Developer 2 branch
            \node[commit] (d2a) at (4,3) {B1};
            \node[commit] (d2b) at (5,3) {B2};
            \node[commit] (d2c) at (6,3) {B3};
            
            % Arrows
            \draw[arrow, gludOrangeLight] (m1) -- (m2);
            \draw[arrow, gludOrangeLight] (m2) -- (m3);
            \draw[arrow, gludOrangeLight] (m3) -- (m4);
            \draw[arrow, gludOrangeLight] (m4) -- (m5);
            
            \draw[arrow, gludPeach] (m2) -- (d1a);
            \draw[arrow, gludOrangeLight] (d1a) -- (d1b);
            \draw[arrow, gludPeach] (d1b) -- (m3);
            
            \draw[arrow, gludPeach] (m3) -- (d2a);
            \draw[arrow, gludOrangeLight] (d2a) -- (d2b);
            \draw[arrow, gludOrangeLight] (d2b) -- (d2c);
            \draw[arrow, gludPeach] (d2c) -- (m5);
            
            % Labels
            \node[branch] at (0,-0.7) {main};
            \node[branch] at (3,2.2) {dev1/feature};
            \node[branch] at (6,3.7) {dev2/bugfix};
        \end{tikzpicture}
    \end{center}
    
    \vspace{0.3cm}
    
    {\scriptsize\color{gludLightGray} Cada desarrollador trabaja en sus propias ramas y las fusiona cuando completa}
\end{frame}

\begin{frame}{Feature Branch: Flujo de Trabajo}
    \step{1}{Crear rama desde main}
    
    \begin{codebox}
        \ttfamily\small\color{gludWhite}
        git switch main\\
        git pull\\
        git switch -c juan/agregar-autenticacion
    \end{codebox}
    
    \pause
    
    \step{2}{Desarrollar y hacer commits}
    \pause
    
    \step{3}{Fusionar a main}
    
    \begin{codebox}
        \ttfamily\small\color{gludWhite}
        git switch main \& git pull\\
        git merge juan/agregar-autenticacion\\
        git push
    \end{codebox}
\end{frame}

\begin{frame}{Feature Branch: Convenciones de Nombres}
    
    \begin{orangebox}[Formato común: desarrollador/descripción]
        \begin{itemize}
            {\color{black}
            \item \texttt{juan/login-page}
            \item \texttt{maria/fix-database-error}
            \item \texttt{carlos/refactor-api}
            }
        \end{itemize}
    \end{orangebox}
    
    \vspace{0.4cm}
    \pause
    
    \textbf{Alternativas:}
    \begin{itemize}
        \item Por ticket: \texttt{JIRA-123-login}, \texttt{issue-45-bug}
        \item Mixto: \texttt{juan/feature/login}
    \end{itemize}

    \pause
    
    \centering
    {\color{gludOrangeLight} \textbf{Importante:} El equipo debe acordar y seguir una convención}
\end{frame}

\begin{frame}{Feature Branch: Ventajas y Desventajas}
    \vspace{0.3cm}
    
    \begin{columns}[T]
        \begin{column}{0.48\textwidth}
            \textbf{\color{gludOrange} Ventajas}
            \begin{itemize}
                \item Muy simple de entender
                \item Ideal para CI/CD
                \item Flexibilidad total
                \item Rápido para equipos pequeños
                \item Menos overhead
            \end{itemize}
        \end{column}
        \pause
        \begin{column}{0.48\textwidth}
            \textbf{\color{gludOrange} Desventajas}
            \begin{itemize}
                \item Sin estructura formal
                \item Propenso a errores humanos
                \item Puede volverse caótico
                \item Difícil gestionar releases
                \item Requiere code reviews estrictas
            \end{itemize}
        \end{column}
    \end{columns}
    
    \vspace{0.5cm}
    \pause
    
    \centering
    {\color{gludOrangeLight} \textbf{Ideal para:} Equipos pequeños, despliegue continuo, proyectos ágiles}
\end{frame}

% SECTION 6: COMPARACIÓN Y BUENAS PRÁCTICAS

\section{Comparación y Buenas Prácticas}

\begin{frame}{¿Cuál Elegir?}
    \vspace{0.3cm}
    
    \begin{columns}[T]
        \begin{column}{0.48\textwidth}
            \begin{darkbox}[Elige Git Flow si:]
                {\color{gludWhite}
                \begin{itemize}
                    \item Tienes releases programados
                    \item Equipo mediano o grande
                    \item Múltiples versiones en paralelo
                    \item Necesitas estabilidad máxima
                    \item Proceso formal de QA
                \end{itemize}
                }
            \end{darkbox}
        \end{column}
        \pause
        \begin{column}{0.48\textwidth}
            \begin{darkbox}[Elige Feature Branch si:]
                {\color{gludWhite}
                \begin{itemize}
                    \item Despliegue continuo
                    \item Equipo pequeño
                    \item Proyecto ágil/startup
                    \item Quieres simplicidad de desarrollo
                    \item Menos burocracia en el flujo de trabajo
                \end{itemize}
                }
            \end{darkbox}
        \end{column}
    \end{columns}
    
    \vspace{0.5cm}
    \pause
    
    \centering
    {\color{gludOrangeLight} No hay una respuesta única --- depende del contexto}
\end{frame}

\begin{frame}{Buenas Prácticas Universales}
    \vspace{0.3cm}
    
    \begin{orangebox}[Aplica a cualquier estrategia]
        \begin{itemize}
            {\color{black}
            \item \textbf{Nombres descriptivos:} \texttt{feature/login-oauth} mejor que \texttt{rama1}
            \item \textbf{Ramas cortas:} Fusionar frecuentemente para evitar conflictos
            \item \textbf{main siempre estable:} No hacer commit directo a main
            \item \textbf{Sincronizar:} Hacer \texttt{git pull} antes de crear ramas
            \item \textbf{Eliminar ramas fusionadas:} Mantener repositorio limpio
            }
        \end{itemize}
    \end{orangebox}
\end{frame}

% SECTION 7: SIGUIENTE PASO

\section{¿Qué sigue?}

\begin{frame}{Fusionar Ramas: Merge}
    \vspace{0.3cm}
    
    \begin{orangebox}[Próxima Clase]
        {\color{black}Hemos visto cómo crear y gestionar ramas, pero ¿cómo las \highlight{unimos}?}
    \end{orangebox}
    
    \vspace{0.4cm}
    \pause
    
    \textbf{En la Clase 5 aprenderemos:}
    \begin{itemize}
        \item \texttt{git merge} --- Fusionar ramas
        \item \texttt{git rebase} --- Reescribir historial
        \item Resolución de conflictos
        \item Fast-forward vs 3-way merge
        \item Estrategias de fusión
        \item Cherry-pick
    \end{itemize}
    
    \vspace{0.4cm}
    
    \centering
    {\color{gludOrangeLight} ¡La verdadera magia de Git viene al combinar ramas!}
\end{frame}

\quoteslide{Branches are cheap and easy, use them early and often.}{Git Best Practices}

\begin{frame}{Comandos Clave}
    \vspace{0.3cm}
    
    \begin{columns}[T]
        \begin{column}{0.48\textwidth}
            \begin{darkbox}[Git Estándar]
                {\color{gludWhite}
                \begin{codebox}
                    \ttfamily\tiny\color{gludWhite}
                    git branch\\
                    git switch -c nombre\\
                    git switch nombre\\
                    git branch -d nombre\\
                    git branch -{-}merged
                \end{codebox}
                }
            \end{darkbox}
        \end{column}
        \begin{column}{0.48\textwidth}
            \begin{darkbox}[git-flow]
                {\color{gludWhite}
                \begin{codebox}
                    \ttfamily\tiny\color{gludWhite}
                    git flow init\\
                    git flow feature start\\
                    git flow feature finish\\
                    git flow hotfix start\\
                    git flow hotfix finish
                \end{codebox}
                }
            \end{darkbox}
        \end{column}
    \end{columns}
\end{frame}

\thankslide{¡Gracias!\\[0.3cm]{\hspace*{1cm}\Large Nos vemos en la próxima clase}}

\end{document}
