\documentclass[aspectratio=169,10pt]{beamer}
\usepackage{../theme}

% METADATA

\title{Ciclo de Vida del Código}
\subtitle{Estados del Código en Git}
\author{GLUD --- Grupo GNU/Linux Universidad Distrital}
\date{Control de Versiones y Desarrollo Colaborativo}

% DOCUMENT

\begin{document}

% Portada
\titleframe[Clase 2 · Ciclo de Vida del Código]

\contentsframe

% SECTION 1: ESTADOS DE ARCHIVOS

\section{Estados de Archivos}

\begin{frame}{El Ciclo de Vida de un Archivo en Git}
    \vspace{0.3cm}
    \centering
    
    \begin{tikzpicture}[scale=0.9]
        % Untracked
        \fill[gludLightGray] (0,0) rectangle (3,2);
        \node[text=gludDark,font=\bfseries] at (1.5,1.3) {Untracked};
        \node[text=gludGray,font=\tiny] at (1.5,0.5) {Sin rastrear};
        
        \pause
        
        % Arrow 1
        \draw[->,thick,gludOrange] (3.2,1) -- (4.5,1);
        \node[text=gludWhite,font=\tiny] at (3.85,1.4) {git add};
        
        % Staged
        \fill[gludOrange] (4.7,0) rectangle (7.7,2);
        \node[text=gludWhite,font=\bfseries] at (6.2,1.3) {Staged};
        \node[text=gludPeach,font=\tiny] at (6.2,0.5) {Preparado};
        
        \pause
        
        % Arrow 2
        \draw[->,thick,gludOrange] (7.9,1) -- (9.2,1);
        \node[text=gludWhite,font=\tiny] at (8.55,1.4) {git commit};
        
        % Committed/Unmodified
        \fill[gludDarker] (9.4,0) rectangle (12.4,2);
        \draw[gludOrange,thick] (9.4,0) rectangle (12.4,2);
        \node[text=gludWhite,font=\bfseries] at (10.9,1.3) {Committed};
        \node[text=gludLightGray,font=\tiny] at (10.9,0.5) {Sin modificar};
        
        \pause
        
        % Arrow 3 (modify) -> curve up to the top of the Modified box
        \draw[->,thick,gludOrange] (10.9,2.2) .. controls (10.9,4.0) and (8.7,4.3) .. (7.9,4.0);
        \node[text=gludWhite,font=\tiny] at (8.8,3.8) {editar archivo};
        
        % Modified
        \fill[gludPeach] (4.7,2.5) rectangle (7.7,4.5);
        \node[text=gludDark,font=\bfseries] at (6.2,3.8) {Modified};
        \node[text=gludGray,font=\tiny] at (6.2,3) {Modificado};
        
        \pause
        
        % Arrow back to staged
        \draw[->,thick,gludOrange] (6.2,2.4) -- (6.2,2.1);
        \node[text=gludWhite,font=\tiny,anchor=west] at (6.5,2.3) {git add};
    \end{tikzpicture}
    
    \vspace{0.4cm}
    
    \begin{columns}[T]
        \begin{column}{0.32\textwidth}
            \centering
            \circled{1} \highlight{Untracked}
            
            {\tiny Archivo nuevo, Git no lo conoce}
        \end{column}
        \begin{column}{0.32\textwidth}
            \centering
            \circled{2} \highlight{Staged}
            
            {\tiny Listo para commit}
        \end{column}
        \begin{column}{0.32\textwidth}
            \centering
            \circled{3} \highlight{Committed}
            
            {\tiny Guardado en historial}
        \end{column}
    \end{columns}
\end{frame}

\begin{frame}{Estado: Untracked}
    
    \begin{orangebox}[Definición]
        Un archivo \highlight{Untracked} es un archivo nuevo que Git aún no está rastreando.
    \end{orangebox}

    \pause
    
    \begin{columns}[T]
        \begin{column}{0.48\textwidth}
            \begin{codebox}[Crear archivo]
                \ttfamily\small\color{gludWhite}
                echo "Hola" > nuevo.txt\\
                git status
            \end{codebox}
            
            \textbf{Características:}
            \begin{itemize}
                \item Git lo detecta pero no lo sigue
                \item Aparece en rojo en \texttt{git status}
                \item No se incluye en commits
            \end{itemize}
        \end{column}
        \pause
        \begin{column}{0.48\textwidth}
            \begin{codebox}[Salida de git status]
                \ttfamily\tiny\color{gludWhite}
                Untracked files:\\
                \hspace{0.5cm}(use "git add..." )\\[4pt]
                \hspace{1cm}{\color{red}nuevo.txt}
            \end{codebox}
            
            \vspace{0.3cm}
            
            {\small\color{gludLightGray} Usa \texttt{git add} para comenzar a rastrearlo}
        \end{column}
    \end{columns}
\end{frame}

\begin{frame}{Estado: Staged}
    
    \begin{orangebox}[Definición]
        Un archivo \highlight{Staged} está en el área de preparación, listo para ser confirmado.
    \end{orangebox}
    
    \pause
    
    \begin{codebox}[Agregar al staging]
        \ttfamily\color{gludWhite}
        git add nuevo.txt\\
        git status
    \end{codebox}
    
    \pause
    
    \begin{codebox}[Salida de git status]
        \ttfamily\tiny\color{gludWhite}
        Changes to be committed:\\
        \hspace{0.5cm}(use "git restore -{}-staged..." to unstage)\\[4pt]
        \hspace{1cm}{\color{green}new file:   nuevo.txt}
    \end{codebox}
    
\end{frame}

\begin{frame}{Estado: Committed / Unmodified}
    
    \begin{orangebox}[Definición]
        Un archivo \highlight{Committed} está guardado en el historial de Git y no ha sido modificado desde el último commit.
    \end{orangebox}
    
    \pause
    
    \begin{codebox}[Crear commit]
        \ttfamily\color{gludWhite}
        git commit -m "feat: agregar archivo nuevo"\\
        git status
    \end{codebox}
    
    \pause
    
    \begin{codebox}[Salida de git status]
        \ttfamily\small\color{gludWhite}
        On branch main\\
        nothing to commit, working tree clean
    \end{codebox}

\end{frame}

\begin{frame}{Estado: Modified}
    
    \begin{orangebox}[Definición]
        Un archivo \highlight{Modified} ha sido editado desde el último commit pero aún no está en staging.
    \end{orangebox}
    
    \pause
    
    \begin{codebox}[Modificar archivo]
        \ttfamily\color{gludWhite}
        echo "Línea nueva" >> nuevo.txt\\
        git status
    \end{codebox}
    
    \pause
    
    \begin{codebox}[Salida de git status]
        \ttfamily\tiny\color{gludWhite}
        Changes not staged for commit:\\
        \hspace{0.5cm}(use "git add..." to update)\\
        \hspace{0.5cm}(use "git restore..." to discard)\\[4pt]
        \hspace{1cm}{\color{red}modified:   nuevo.txt}
    \end{codebox}
    
\end{frame}

\begin{frame}{Comandos de Control de Estado}
    \vspace{0.3cm}
    
    \begin{columns}[T]
        \begin{column}{0.48\textwidth}
            \begin{codebox}[Ver estado]
                \ttfamily\small\color{gludWhite}
                git status\\[4pt]
                git status -s
            \end{codebox}
            
            \vspace{0.3cm}
            
            \begin{codebox}[Sacar de staging]
                \ttfamily\small\color{gludWhite}
                git restore -{}-staged\\
                \hspace{0.5cm}archivo.txt
            \end{codebox}
        \end{column}
        \pause
        \begin{column}{0.48\textwidth}
            \begin{codebox}[Descartar cambios]
                \ttfamily\small\color{gludWhite}
                git restore archivo.txt
            \end{codebox}
            
            \vspace{0.3cm}
            
            \begin{alertbox}[Cuidado]
                \texttt{git restore} sin \texttt{-{}-staged} descarta cambios permanentemente
            \end{alertbox}
        \end{column}
    \end{columns}
\end{frame}

% SECTION 2: CONVENTIONAL COMMITS

\section{Conventional Commits}

\begin{frame}{¿Qué son los Conventional Commits?}
    
    \begin{orangebox}[Definición]
        \highlight{Conventional Commits} es una convención para escribir mensajes de commit estructurados y significativos.
    \end{orangebox}

    \pause
    
    \begin{codebox}[Estructura]
        \ttfamily\small\color{gludWhite}
        <tipo>[ámbito opcional]: <descripción>\\[6pt]
        [cuerpo opcional]\\[6pt]
        [nota(s) al pie opcional(es)]
    \end{codebox}
    
    \pause
    
    \centering
    \textbf{Ventajas:}
    
    \vspace{0.2cm}
    
    \begin{columns}[T]
        \begin{column}{0.32\textwidth}
            \centering
            {\small Historial legible}
        \end{column}
        \begin{column}{0.32\textwidth}
            \centering
            {\small Automatización}
        \end{column}
        \begin{column}{0.32\textwidth}
            \centering
            {\small Comunicación clara}
        \end{column}
    \end{columns}
\end{frame}

\begin{frame}{Tipos de Commit}
    \vspace{0.3cm}
    
    \begin{columns}[T]
        \begin{column}{0.48\textwidth}
            \textbf{Cambios principales:}
            \begin{itemize}
                \item \highlight{feat}: Nueva funcionalidad
                \item \highlight{fix}: Corrección de bug
                \item \highlight{refactor}: Cambio de código sin nueva función
                \item \highlight{perf}: Mejora de rendimiento
            \end{itemize}
        \end{column}
        \pause
        \begin{column}{0.48\textwidth}
            \textbf{Cambios auxiliares:}
            \begin{itemize}
                \item \highlight{docs}: Documentación
                \item \highlight{style}: Formato (espacios, etc.)
                \item \highlight{test}: Agregar/modificar tests
                \item \highlight{chore}: Tareas de mantenimiento
                \item \highlight{build}: Cambios en build
                \item \highlight{ci}: Integración continua
            \end{itemize}
        \end{column}
    \end{columns}
\end{frame}

\begin{frame}{Ejemplos de Conventional Commits}
    \vspace{0.3cm}
    
    \begin{codebox}[feat: Nueva funcionalidad]
        \ttfamily\small\color{gludWhite}
        git commit -m "feat: agregar validación de email"\\
        git commit -m "feat(auth): implementar login con JWT"
    \end{codebox}
    
    \pause
    
    \begin{codebox}[fix: Corrección]
        \ttfamily\small\color{gludWhite}
        git commit -m "fix: corregir cálculo de total"\\
        git commit -m "fix(api): resolver error 500 en endpoint"
    \end{codebox}
    
    \pause
    
    \begin{codebox}[docs: Documentación]
        \ttfamily\small\color{gludWhite}
        git commit -m "docs: actualizar README con instrucciones"\\
        git commit -m "docs(api): documentar endpoint de usuarios"
    \end{codebox}
\end{frame}

\begin{frame}{Más Ejemplos}
    \vspace{0.3cm}
    
    \begin{codebox}[refactor: Refactorización]
        \ttfamily\small\color{gludWhite}
        git commit -m "refactor: simplificar lógica de validación"\\
        git commit -m "refactor(service): extraer clase helper"
    \end{codebox}
    
    \pause
    
    \begin{codebox}[chore: Tareas de mantenimiento]
        \ttfamily\small\color{gludWhite}
        git commit -m "chore: actualizar dependencias"\\
        git commit -m "chore: configurar .gitignore"
    \end{codebox}
    
    \pause
    
    \begin{codebox}[test: Pruebas]
        \ttfamily\small\color{gludWhite}
        git commit -m "test: agregar tests unitarios"\\
        git commit -m "test(user): validar creación de usuario"
    \end{codebox}
\end{frame}

\begin{frame}{Commits con Breaking Changes}
    \vspace{0.3cm}
    
    \begin{orangebox}[Breaking Change]
        Un \highlight{breaking change} es un cambio que rompe la compatibilidad con versiones anteriores.
    \end{orangebox}
    
    \pause
    
    \begin{codebox}[Sintaxis con !]
        \ttfamily\small\color{gludWhite}
        git commit -m "feat!: cambiar estructura de API"\\
        git commit -m "refactor(core)!: renombrar métodos principales"
    \end{codebox}
    
    \pause
    
    \begin{codebox}[Con nota al pie]
        \ttfamily\tiny\color{gludWhite}
        git commit -m "feat: cambiar endpoint de login\\[4pt]
        BREAKING CHANGE: el endpoint cambió de /auth a /api/v2/auth"
    \end{codebox}
\end{frame}

\quoteslide{A commit message should tell a story about what changed and why.}{Linus Torvalds}

% SECTION 3: ESTRUCTURA DE PROYECTO

\section{Estructura de Proyecto}

\begin{frame}{Proyecto Maven Estándar}
    \vspace{0.3cm}
    
    \begin{columns}[T]
        \begin{column}{0.48\textwidth}
                    \begin{codebox}[Estructura]
                        \ttfamily\tiny\color{gludWhite}
                        mi-proyecto/\\
                        - pom.xml\\
                        - README.md\\
                        - .gitignore\\
                        - src/\\
                        \hspace*{0.25cm}- main/\\
                        \hspace*{0.5cm}- java/\\
                        \hspace*{0.75cm}- com/ejemplo/\\
                        \hspace*{1cm}- Main.java\\
                        \hspace*{1cm}- modelo/\\
                        \hspace*{0.25cm}- resources/\\
                        \hspace*{0.25cm}- test/\\
                        \hspace*{0.5cm}- java/
                    \end{codebox}
                \end{column}
                \pause
                \begin{column}{0.48\textwidth}
                    \textbf{Archivos clave:}
                    \begin{itemize}
                        \item \inlinecode{pom.xml}: Configuración Maven
                        \item \inlinecode{README.md}: Documentación
                        \item \inlinecode{.gitignore}: Archivos a ignorar
                        \item \inlinecode{src/main/java}: Código fuente
                        \item \inlinecode{src/test/java}: Pruebas
                    \end{itemize}
                \end{column}
    \end{columns}
\end{frame}

\begin{frame}{Proyecto Python con UV}
    \vspace{0.3cm}
    
    \begin{columns}[T]
        \begin{column}{0.48\textwidth}
                    \begin{codebox}[Estructura]
                        \ttfamily\tiny\color{gludWhite}
                        mi-proyecto/\\
                        - pyproject.toml\\
                        - README.md\\
                        - .gitignore\\
                        - .python-version\\
                        - src/\\
                        \hspace*{0.25cm}- mi\_proyecto/\\
                        \hspace*{0.5cm}- \_\_init\_\_.py\\
                        \hspace*{0.5cm}- main.py\\
                        \hspace*{0.5cm}- utils.py\\
                        - tests/\\
                        \hspace*{0.25cm}- test\_main.py\\
                        \hspace*{0.25cm}- test\_utils.py
                    \end{codebox}
                \end{column}
                \pause
                \begin{column}{0.48\textwidth}
                    \textbf{Archivos clave:}
                    \begin{itemize}
                        \item \inlinecode{pyproject.toml}: Configuración UV y dependencias
                        \item \inlinecode{README.md}: Documentación
                        \item \inlinecode{.gitignore}: Archivos a ignorar
                        \item \inlinecode{.python-version}: Versión de Python
                        \item \inlinecode{src/}: Código fuente
                        \item \inlinecode{tests/}: Pruebas unitarias
                    \end{itemize}
                \end{column}
    \end{columns}
\end{frame}

\begin{frame}{Crear Proyecto con UV}
    \vspace{0.3cm}
    
    \step{1}{Crear proyecto desde cero}
    
    \begin{codebox}
        \ttfamily\small\color{gludWhite}
        uv init mi-proyecto\\
        cd mi-proyecto
    \end{codebox}
    
    \pause
    
    \step{2}{UV crea automáticamente}
    
    \begin{itemize}
        \item Estructura de directorios completa
        \item \texttt{pyproject.toml} configurado
        \item Archivo \texttt{main.py} de ejemplo
        \item Archivo \texttt{.python-version}
    \end{itemize}
    
    \pause
    
    \step{3}{Instalar dependencias}
    
    \begin{codebox}
        \ttfamily\small\color{gludWhite}
        uv sync
    \end{codebox}
\end{frame}

\begin{frame}{.gitignore para Python/UV}
    
    \begin{codebox}[.gitignore]
        \ttfamily\small\color{gludWhite}
        \# Entorno virtual\\
        .venv/\\
        venv/\\
        \_\_pycache\_\_/\\
        *.pyc\\
        *.pyo\\
        dist/\\
        build/\\
        *.egg-info/\\
        *.log\\
        \# Variables de entorno\\
        .env\\
        .env.local
    \end{codebox}
\end{frame}

\begin{frame}{Crear Estructura con Maven}
    \vspace{0.3cm}
    
    \step{1}{Crear proyecto desde arqueotipo}
    
    \begin{codebox}
        \ttfamily\tiny\color{gludWhite}
        mvn archetype:generate\\
        \hspace{0.5cm}-DgroupId=com.ejemplo\\
        \hspace{0.5cm}-DartifactId=mi-proyecto\\
        \hspace{0.5cm}-DarchetypeArtifactId=maven-archetype-quickstart\\
        \hspace{0.5cm}-DinteractiveMode=false
    \end{codebox}
    
    \pause
    
    \step{2}{Maven crea automáticamente}
    
    \begin{itemize}
        \item Estructura de directorios completa
        \item \texttt{pom.xml} configurado
        \item Clase \texttt{App.java} de ejemplo
        \item Clase de test \texttt{AppTest.java}
    \end{itemize}
\end{frame}

\begin{frame}{El archivo pom.xml}
    \vspace{0.3cm}
    
    \begin{codebox}[pom.xml básico]
        \ttfamily\tiny\color{gludWhite}
        <project>\\
        \hspace{0.5cm}<modelVersion>4.0.0</modelVersion>\\
        \hspace{0.5cm}<groupId>com.ejemplo</groupId>\\
        \hspace{0.5cm}<artifactId>mi-proyecto</artifactId>\\
        \hspace{0.5cm}<version>1.0-SNAPSHOT</version>\\[6pt]
        \hspace{0.5cm}<properties>\\
        \hspace{1cm}<maven.compiler.source>21</maven.compiler.source>\\
        \hspace{1cm}<maven.compiler.target>21</maven.compiler.target>\\
        \hspace{0.5cm}</properties>\\[6pt]
        \hspace{0.5cm}<dependencies>\\
        \hspace{1cm}<!-- tus dependencias -->\\
        \hspace{0.5cm}</dependencies>\\
        </project>
    \end{codebox}
\end{frame}

\begin{frame}{README.md: Carta de Presentación}
    \vspace{0.3cm}
    
    \begin{orangebox}[Importancia]
        El \highlight{README.md} es lo primero que ven otros desarrolladores. Debe ser claro y completo.
    \end{orangebox}
    
    \vspace{0.4cm}
    \pause
    
    \textbf{Secciones recomendadas:}
    \begin{itemize}
        \item Título y descripción del proyecto
        \item Requisitos (Java 21, Maven 3.X)
        \item Instrucciones de instalación
        \item Cómo ejecutar el proyecto
        \item Ejemplos de uso
        \item Licencia
        \item Contacto/contribución
    \end{itemize}
\end{frame}

\begin{frame}{Ejemplo de README.md}
    \vspace{0.3cm}
    
    \begin{codebox}[README.md]
        \ttfamily\tiny\color{gludWhite}
        \# Mi Proyecto\\[4pt]
        Descripción breve del proyecto.\\[6pt]
        \#\# Requisitos\\
        - Java 21+\\
        - Maven 3.8+\\[6pt]
        \#\# Instalación\\
        ```bash\\
        git clone https://github.com/usuario/mi-proyecto.git\\
        cd mi-proyecto\\
        mvn clean install\\
        ```\\[6pt]
        \#\# Uso\\
        ```bash\\
        mvn exec:java -Dexec.mainClass="com.ejemplo.Main"\\
        ```
    \end{codebox}
\end{frame}

% SECTION 4: GITIGNORE

\section{Configuración de .gitignore}

\begin{frame}{¿Por qué usar .gitignore?}
    \vspace{0.3cm}
    
    \begin{orangebox}[Propósito]
        El archivo \highlight{.gitignore} indica a Git qué archivos \textbf{no} debe rastrear.
    \end{orangebox}
    
    \vspace{0.4cm}
    \pause
    
    \textbf{Archivos a ignorar:}
    \begin{itemize}
        \item Archivos compilados (\texttt{*.class}, \texttt{target/})
        \item Archivos de configuración local (\texttt{.vscode/}, \texttt{.idea/})
        \item Dependencias descargadas (\texttt{node\_modules/})
        \item Archivos de sistema (\texttt{.DS\_Store}, \texttt{Thumbs.db})
        \item Logs y temporales (\texttt{*.log}, \texttt{*.tmp})
        \item Credenciales y secretos (\texttt{.env}, \texttt{secrets.yml})
    \end{itemize}
    
    \vspace{0.3cm}
    
    \centering
    {\color{gludOrangeLight} Nunca subas contraseñas, tokens o API keys}
\end{frame}

\begin{frame}{.gitignore para Java/Maven}
    
    \begin{codebox}[.gitignore]
        \ttfamily\small\color{gludWhite}
        \# Archivos compilados\\
        *.class\\
        *.jar\\
        *.war\\[6pt]
        \# Directorio de Maven\\
        target/\\[6pt]
        \# IDEs\\
        .idea/\\
        .vscode/\\[6pt]
        \# Logs\\
        *.log
    \end{codebox}
\end{frame}

\begin{frame}{Usando gitignore.io}
    
    \begin{orangebox}[¿Qué es gitignore.io?]
        \highlight{gitignore.io} es un servicio que genera archivos .gitignore automáticamente según tu stack tecnológico.
    \end{orangebox}
    
    \vspace{0.4cm}
    \pause
    
    \step{1}{Visita \url{https://www.toptal.com/developers/gitignore}}
    
    \vspace{0.2cm}
    \pause
    
    \step{2}{Escribe: \texttt{Java}, \texttt{Maven}, \texttt{IntelliJ}, \texttt{VisualStudioCode}}
    
    \vspace{0.2cm}
    \pause
    
    \step{3}{Copia el contenido generado a tu \texttt{.gitignore}}
    
    \vspace{0.3cm}
    
    \centering
    {\color{gludLightGray} También puedes usar la API o CLI: \texttt{gi java,maven > .gitignore}}
\end{frame}

\begin{frame}{Aplicar .gitignore a Archivos Rastreados}
    
    \begin{alertbox}[Problema]
        Si ya agregaste archivos a Git antes de crear .gitignore, seguirán siendo rastreados.
    \end{alertbox}
    
    \vspace{0.4cm}
    \pause
    
    \begin{codebox}[Solución]
        \ttfamily\small\color{gludWhite}
        \# Remover del índice (sin borrar del disco)\\
        git rm -{}-cached -r target/\\[6pt]
        \# Agregar .gitignore\\
        git add .gitignore\\[6pt]
        \# Commit\\
        git commit -m "chore: configurar .gitignore para Java/Maven"
    \end{codebox}
    
    \vspace{0.3cm}
    
    \centering
    {\color{gludOrangeLight} Ahora \texttt{target/} será ignorado en futuros commits}
\end{frame}

\begin{frame}{Patrones Comunes en .gitignore}
    \vspace{0.3cm}
    
    \begin{columns}[T]
        \begin{column}{0.48\textwidth}
            \begin{codebox}[Patrones]
                \ttfamily\small\color{gludWhite}
                \# Archivo específico\\
                config.local\\[4pt]
                \# Extensión\\
                *.log\\[4pt]
                \# Directorio\\
                build/\\[4pt]
                \# Todos menos uno\\
                *.jar\\
                !lib/importante.jar
            \end{codebox}
        \end{column}
        \pause
        \begin{column}{0.48\textwidth}
            \begin{codebox}[Más patrones]
                \ttfamily\small\color{gludWhite}
                \# Cualquier nivel\\
                **/temp/\\[4pt]
                \# En raíz solamente\\
                /TODO\\[4pt]
                \# Comentario\\
                \# esto es un comentario\\[4pt]
                \# Negación\\
                !importante.txt
            \end{codebox}
        \end{column}
    \end{columns}
\end{frame}

% SECTION 6: BUENAS PRÁCTICAS

\section{Buenas Prácticas}

\begin{frame}{Commits Atómicos}
    \vspace{0.3cm}
    
    \begin{orangebox}[Concepto]
        Un \highlight{commit atómico} contiene un único cambio lógico y completo.
    \end{orangebox}
    
    \vspace{0.4cm}
    \pause
    
    \begin{columns}[T]
        \begin{column}{0.48\textwidth}
            \begin{darkbox}[Bien]
                \begin{itemize}
                    \item Un bug, un commit
                    \item Una feature, uno o más commits
                    \item Cada commit compila
                    \item Mensajes descriptivos
                \end{itemize}
            \end{darkbox}
        \end{column}
        \pause
        \begin{column}{0.48\textwidth}
            \begin{orangebox}[Mal]
                \begin{itemize}
                    \item {\color{black}Múltiples features en un commit}
                    \item {\color{black}Commit con código roto}
                    \item {\color{black}Mensajes vagos: "fix", "cambios"}
                    \item {\color{black}Commits gigantes}
                \end{itemize}
            \end{orangebox}
        \end{column}
    \end{columns}
\end{frame}

\begin{frame}{Reglas de Oro para Commits}
    \vspace{0.3cm}
    
    \begin{enumerate}
        \item \textbf{Commit frecuentemente} --- No esperes al final del día
        \pause
        \item \textbf{Usa Conventional Commits} --- Historial legible y profesional
        \pause
        \item \textbf{Revisa antes de commitear} --- Usa \texttt{git diff} y \texttt{git status}
        \pause
        \item \textbf{No subas archivos generados} --- Usa .gitignore correctamente
        \pause
        \item \textbf{Nunca subas secretos} --- Contraseñas, tokens, API keys fuera
        \pause
        \item \textbf{Escribe mensajes claros} --- Piensa en tus compañeros
    \end{enumerate}
\end{frame}

\begin{frame}{Checklist Antes de Commitear}
    \vspace{0.3cm}
    
    \centering
    
    \begin{tikzpicture}
        \node[draw=gludOrange,thick,fill=gludDark,text=gludWhite,rounded corners,text width=10cm,align=left,font=\small] {%
            \textbf{Antes de hacer commit:}\\[8pt]
            $\square$ ¿El código compila? (\texttt{mvn compile})\\
            $\square$ ¿Revisé los cambios? (\texttt{git diff})\\
            $\square$ ¿Agregué solo lo necesario? (\texttt{git status})\\
            $\square$ ¿El mensaje es descriptivo?\\
            $\square$ ¿Sigo Conventional Commits?\\
            $\square$ ¿No hay archivos generados?\\
            $\square$ ¿No hay secretos/contraseñas?
        };
    \end{tikzpicture}
\end{frame}

\quoteslide{Commit early, commit often.}{Git Best Practices}

% SECTION 7: RESUMEN

\section{Resumen}

\begin{frame}{Lo que Aprendimos}
    \vspace{0.3cm}
    
    \begin{columns}[T]
        \begin{column}{0.48\textwidth}
            \textbf{Estados de archivos:}
            \begin{itemize}
                \item Untracked
                \item Staged
                \item Committed / Unmodified
                \item Modified
            \end{itemize}
            
            \vspace{0.3cm}
            
            \textbf{Conventional Commits:}
            \begin{itemize}
                \item Tipos: feat, fix, docs, etc.
                \item Estructura clara
                \item Breaking changes
            \end{itemize}
        \end{column}
        \begin{column}{0.48\textwidth}
            \textbf{Estructura de proyecto:}
            \begin{itemize}
                \item Maven estándar
                \item pom.xml
                \item README.md
            \end{itemize}
            
            \vspace{0.3cm}
            
            \textbf{.gitignore:}
            \begin{itemize}
                \item Ignorar compilados
                \item Ignorar IDEs
                \item Usar gitignore.io
                \item Patrones comunes
            \end{itemize}
        \end{column}
    \end{columns}
\end{frame}

\begin{frame}{Comandos Clave}
    \vspace{0.3cm}
    
    \begin{codebox}
        \ttfamily\small\color{gludWhite}
        \# Estados\\
        git status\\[6pt]
        \# Staging\\
        git add archivo.txt\\
        git restore -{}-staged archivo.txt\\[6pt]
        \# Commits\\
        git commit -m "feat: nueva funcionalidad"\\[6pt]
        \# Diferencias\\
        git diff\\
        git diff -{}-staged\\[6pt]
        \# Descartar cambios\\
        git restore archivo.txt
    \end{codebox}
\end{frame}

\begin{frame}{Recursos}
    \vspace{0.5cm}
    
    \begin{columns}[T]
        \begin{column}{0.48\textwidth}
            \textbf{Conventional Commits:}
            \begin{itemize}
                \item \url{https://www.conventionalcommits.org}
                \item Especificación completa
                \item Ejemplos y guías
            \end{itemize}
        \end{column}
        \begin{column}{0.48\textwidth}
            \textbf{.gitignore:}
            \begin{itemize}
                \item \url{https://www.toptal.com/developers/gitignore}
                \item Templates oficiales
                \item GitHub gitignore repo
            \end{itemize}
        \end{column}
    \end{columns}
    
    \vspace{0.5cm}
    
    \textbf{Maven:}
    \begin{itemize}
        \item \url{https://maven.apache.org/guides/}
        \item Maven in 5 Minutes
        \item POM Reference
    \end{itemize}
\end{frame}

% Next class preview

\begin{frame}[plain]
\setbeamercolor{background canvas}{bg=gludPeach}
\vspace{1.5cm}
\centering

{\Huge\color{gludOrange}\bfseries Próxima Clase}

\vspace{0.5cm}

{\Large\color{gludDark} Viajes en el Tiempo}

\vspace{1cm}

{\large\color{gludWhite} Historial (git log). Diferencias (git diff avanzado). Restauración (checkout/restore). Etiquetas (tags) para versiones.}
\end{frame}

% Despedida

\thankslide{¡Gracias!\\[0.3cm]{\hspace*{1cm}\Large Nos vemos en la próxima clase}}

\end{document}