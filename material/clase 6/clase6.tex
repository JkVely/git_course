\documentclass[aspectratio=169,10pt]{beamer}
\usepackage{../theme}
\usepackage{colortbl}

% METADATA

\title{Flujos Remotos}
\subtitle{GitHub, GitLab y el Mundo Exterior}
\author{GLUD --- Grupo GNU/Linux Universidad Distrital}
\date{Control de Versiones y Desarrollo Colaborativo}

% DOCUMENT

\begin{document}

% Portada
\titleframe[Clase 6 · Flujos Remotos]

\contentsframe

% ============================================================================
% SECTION 1: INTRODUCCIÓN
% ============================================================================

\section{Del Local al Remoto}

\begin{frame}{Git es Distribuido}
    \vspace{0.3cm}
    
    \begin{orangebox}[Recordatorio]
        Git es un sistema de control de versiones \highlight{distribuido}. Cada clon es un repositorio completo.
    \end{orangebox}
    
    \vspace{0.4cm}
    \pause
    
    \begin{columns}[T]
        \begin{column}{0.48\textwidth}
            \textbf{Hasta ahora:}
            \begin{itemize}
                \item Todo en tu máquina local
                \item Commits, ramas, merges
                \item Sin conexión a internet
                \item Solo tú tienes acceso
            \end{itemize}
        \end{column}
        \pause
        \begin{column}{0.48\textwidth}
            \textbf{Ahora:}
            \begin{itemize}
                \item Conectar con servidores remotos
                \item Compartir código con el equipo
                \item Respaldo en la nube
                \item Colaboración real
            \end{itemize}
        \end{column}
    \end{columns}
\end{frame}

\begin{frame}{GitHub vs GitLab vs Otros}
    \vspace{0.3cm}
    
    \begin{columns}[T]
        \begin{column}{0.32\textwidth}
            \begin{darkbox}[GitHub]
                {\color{gludWhite}
                \begin{itemize}
                    \item El más popular
                    \item Propiedad de Microsoft
                    \item GitHub Actions
                    \item Comunidad enorme
                \end{itemize}
                }
            \end{darkbox}
        \end{column}
        \begin{column}{0.32\textwidth}
            \begin{darkbox}[GitLab]
                {\color{gludWhite}
                \begin{itemize}
                    \item Open source
                    \item Self-hosted posible
                    \item CI/CD integrado
                    \item Todo en uno
                \end{itemize}
                }
            \end{darkbox}
        \end{column}
        \begin{column}{0.32\textwidth}
            \begin{darkbox}[Otros]
                {\color{gludWhite}
                \begin{itemize}
                    \item Bitbucket (Atlassian)
                    \item Gitea (self-hosted)
                    \item Azure DevOps
                    \item Codeberg
                \end{itemize}
                }
            \end{darkbox}
        \end{column}
    \end{columns}
    
    \vspace{0.4cm}
    \pause
    
    \centering
    {\color{gludOrangeLight} Los comandos de Git son los mismos --- solo cambia la interfaz web}
\end{frame}

% ============================================================================
% SECTION 2: AUTENTICACIÓN SSH
% ============================================================================

\section{Autenticación SSH}

\begin{frame}{¿Por Qué SSH?}
    \vspace{0.3cm}
    
    \begin{columns}[T]
        \begin{column}{0.48\textwidth}
            \begin{darkbox}[HTTPS]
                {\color{gludWhite}
                \textbf{URL:} \texttt{https://github.com/user/repo.git}
                
                \vspace{0.2cm}
                
                \begin{itemize}
                    \item Pide usuario/contraseña
                    \item Necesita token de acceso
                    \item Más fácil de configurar
                    \item Puede ser tedioso
                \end{itemize}
                }
            \end{darkbox}
        \end{column}
        \pause
        \begin{column}{0.48\textwidth}
            \begin{darkbox}[SSH]
                {\color{gludWhite}
                \textbf{URL:} \texttt{git@github.com:user/repo.git}
                
                \vspace{0.2cm}
                
                \begin{itemize}
                    \item Usa clave pública/privada
                    \item Sin contraseñas repetidas
                    \item Más seguro
                    \item Configuración inicial
                \end{itemize}
                }
            \end{darkbox}
        \end{column}
    \end{columns}
    
    \vspace{0.4cm}
    
    \centering
    {\color{gludOrangeLight} Recomendación: Usar SSH para desarrollo profesional}
\end{frame}

\begin{frame}{Criptografía de Clave Pública}
    \vspace{0.3cm}
    
    \begin{center}
        \begin{tikzpicture}[
            box/.style={rectangle, rounded corners, fill=gludDarker, text=gludWhite, font=\small, inner sep=10pt, minimum width=3cm, text width=2.8cm, align=center},
            arrow/.style={->, thick, gludOrangeLight}
        ]
            \node[box] (priv) at (0,0) {Clave Privada\\{\tiny\ttfamily id\_ed25519}\\{\scriptsize (SECRETA)}};
            \node[box] (pub) at (5,0) {Clave Pública\\{\tiny\ttfamily id\_ed25519.pub}\\{\scriptsize (Compartible)}};
            \node[box, fill=gludOrange] (server) at (10,0) {GitHub/GitLab\\{\scriptsize (Tu clave pública)}};
            
            \draw[arrow] (priv) -- node[above, font=\tiny] {genera} (pub);
            \draw[arrow] (pub) -- node[above, font=\tiny] {sube} (server);
            
            \node[text=gludLightGray, font=\scriptsize] at (0,-1.5) {Nunca compartas};
            \node[text=gludLightGray, font=\scriptsize] at (5,-1.5) {Puedes compartir};
        \end{tikzpicture}
    \end{center}
    
    \pause
    
    \vspace{0.3cm}
    
    \begin{orangebox}[Analogía]
        {\color{black}La clave pública es como un candado que regalas. Solo tú tienes la llave (clave privada) para abrirlo.}
    \end{orangebox}
\end{frame}

\begin{frame}{Generar Par de Claves SSH}
    
    \begin{codebox}[Generar clave (algoritmo moderno)]
        \ttfamily\color{gludWhite}
        ssh-keygen -t ed25519 -C "tu@email.com"
    \end{codebox}
    
    \pause
    
    \begin{codebox}[Preguntas interactivas]
        \ttfamily\tiny\color{gludWhite}
        Enter file in which to save the key (/home/user/.ssh/id\_ed25519): [Enter]\\
        Enter passphrase (empty for no passphrase): [opcional]\\
        Enter same passphrase again: [opcional]
    \end{codebox}
    
    \pause
    
    \begin{alertbox}[Passphrase]
        {\color{black}La passphrase es una capa extra de seguridad. Si la usas, deberás ingresarla cada vez (o usar ssh-agent).}
    \end{alertbox}
\end{frame}

\begin{frame}{Ubicación de las Claves}
    \vspace{0.3cm}
    
    \begin{codebox}[Las claves se guardan en:]
        \ttfamily\small\color{gludWhite}
        \# Linux/macOS\\
        \textasciitilde/.ssh/id\_ed25519\ \ \ \ \ \# Clave PRIVADA\\
        \textasciitilde/.ssh/id\_ed25519.pub\ \# Clave PÚBLICA\\
    \end{codebox}
    
    \vspace{0.3cm}
    \pause
    
    \begin{codebox}[Ver tu clave pública]
        \ttfamily\color{gludWhite}
        cat \textasciitilde/.ssh/id\_ed25519.pub
    \end{codebox}
    
    \vspace{0.2cm}
    
    {\scriptsize\color{gludLightGray} Salida: \texttt{ssh-ed25519 AAAAC3NzaC1lZDI... tu@email.com}}
\end{frame}

\begin{frame}{Agregar Clave a GitHub}
    \vspace{0.3cm}
    
    \step{1}{Copiar tu clave pública}
    
    \begin{codebox}
        \ttfamily\small\color{gludWhite}
        \# Linux\\
        cat \textasciitilde/.ssh/id\_ed25519.pub | xclip -selection clipboard\\
    \end{codebox}
    
    \pause
    
    \step{2}{En GitHub/GitLab}
    
    {\small Settings → SSH and GPG Keys → New SSH Key → Pegar → Add}
\end{frame}
 
\begin{frame}{Verificar Conexión SSH}
    
    \begin{codebox}[Probar conexión con GitHub]
        \ttfamily\color{gludWhite}
        ssh -T git@github.com
    \end{codebox}
    
    \pause
    
    \begin{codebox}[Respuesta exitosa]
        \ttfamily\small\color{gludWhite}
        Hi username! You've successfully authenticated, but GitHub\\
        does not provide shell access.
    \end{codebox}
    
    \pause
    
    \begin{codebox}[Probar conexión con GitLab]
        \ttfamily\color{gludWhite}
        ssh -T git@gitlab.com
    \end{codebox}
    
    
    {\color{gludOrangeLight} Si funciona, ya puedes clonar y pushear sin contraseña}
\end{frame}

% ============================================================================
% SECTION 3: REPOSITORIOS REMOTOS
% ============================================================================

\section{Repositorios Remotos}

\begin{frame}{Conectar Local con Remoto}
    \vspace{0.3cm}
    
    \textbf{Caso 1: Ya tienes un repo local}
    
    \begin{codebox}
        \ttfamily\small\color{gludWhite}
        git remote add origin git@github.com:usuario/proyecto.git\\
        git branch -M main\\
        git push -u origin main
    \end{codebox}
    
    \vspace{0.4cm}
    \pause
    
    \textbf{Caso 2: Empezar desde cero}
    
    \begin{codebox}
        \ttfamily\small\color{gludWhite}
        git clone git@github.com:usuario/proyecto.git\\
        cd proyecto
    \end{codebox}
\end{frame}

\begin{frame}{El Concepto de ``Origin''}
    \vspace{0.3cm}
    
    \begin{orangebox}[Definición]
        \texttt{origin} es el nombre por defecto del repositorio remoto principal. Es solo un \highlight{alias}.
    \end{orangebox}
    
    \vspace{0.4cm}
    \pause
    
    \begin{codebox}[Ver remotos configurados]
        \ttfamily\color{gludWhite}
        git remote -v
    \end{codebox}
    
    \pause
    
    \begin{codebox}[Salida típica]
        \ttfamily\small\color{gludWhite}
        origin\ \ git@github.com:usuario/proyecto.git (fetch)\\
        origin\ \ git@github.com:usuario/proyecto.git (push)
    \end{codebox}
    
    \vspace{0.3cm}
    
    {\color{gludLightGray}\scriptsize Puedes tener múltiples remotos con diferentes nombres}
\end{frame}

\begin{frame}{Origin vs Upstream}
    \vspace{0.3cm}
    
    \begin{columns}[T]
        \begin{column}{0.48\textwidth}
            \begin{darkbox}[origin]
                {\color{gludWhite}
                \textbf{Tu repositorio}
                
                \vspace{0.2cm}
                
                \begin{itemize}
                    \item Tu fork o repo principal
                    \item Donde haces push
                    \item Tienes permisos de escritura
                \end{itemize}
                }
            \end{darkbox}
        \end{column}
        \pause
        \begin{column}{0.48\textwidth}
            \begin{darkbox}[upstream]
                {\color{gludWhite}
                \textbf{Repositorio original}
                
                \vspace{0.2cm}
                
                \begin{itemize}
                    \item El proyecto original
                    \item De donde haces fetch
                    \item Solo lectura (normalmente)
                \end{itemize}
                }
            \end{darkbox}
        \end{column}
    \end{columns}
    
    \vspace{0.4cm}
    \pause
    
    \begin{codebox}[Agregar upstream]
        \ttfamily\small\color{gludWhite}
        git remote add upstream git@github.com:original/proyecto.git
    \end{codebox}
\end{frame}

% ============================================================================
% SECTION 5: COLABORADORES
% ============================================================================

\section{Gestión de Colaboradores}

\begin{frame}{Roles en GitHub}
    \vspace{0.3cm}
    
    \begin{columns}[T]
        \begin{column}{0.48\textwidth}
            \textbf{\color{gludOrange} Repositorio Personal:}
            \begin{itemize}
                \item Owner (tú)
                \item Collaborators (invitados)
            \end{itemize}
            
            \vspace{0.3cm}
            
            {\scriptsize Solo dos niveles: dueño o colaborador con acceso completo}
        \end{column}
        \pause
        \begin{column}{0.48\textwidth}
            \textbf{\color{gludOrange} Organización:}
            \begin{itemize}
                \item Owner
                \item Admin
                \item Maintainer
                \item Write
                \item Triage
                \item Read
            \end{itemize}
        \end{column}
    \end{columns}
    
    \vspace{0.4cm}
    
    {\color{gludLightGray}\scriptsize Para proyectos grandes, usar Organizaciones permite control granular}
\end{frame}

% ============================================================================
% SECTION 6: CONFIGURACIÓN DEL REPOSITORIO
% ============================================================================

\section{Configuración del Repositorio}

\begin{frame}{Settings Esenciales (GitHub)}
    \vspace{0.3cm}
    
    \textbf{Repository → Settings:}
    
    \vspace{0.3cm}
    
    \begin{itemize}
        \item \textbf{General:} Nombre, descripción, visibilidad
        \item \textbf{Default branch:} Cambiar rama principal
        \item \textbf{Features:} Wikis, Issues, Projects, Discussions
        \item \textbf{Danger Zone:} Archivar, transferir, eliminar
    \end{itemize}
    
    \vspace{0.4cm}
    \pause
    
    \begin{orangebox}[Cambiar rama por defecto]
        {\color{black}Settings → General → Default Branch → Cambiar de \texttt{master} a \texttt{main}}
    \end{orangebox}
\end{frame}

\begin{frame}{Settings Esenciales (GitLab)}
    \vspace{0.3cm}
    
    \textbf{Project → Settings → General:}
    
    \vspace{0.3cm}
    
    \begin{itemize}
        \item \textbf{Naming:} Nombre y descripción
        \item \textbf{Visibility:} Private, Internal, Public
        \item \textbf{Advanced:} Transfer, archive, delete
    \end{itemize}
    
    \vspace{0.3cm}
    \pause
    
    \textbf{Project → Settings → Repository:}
    
    \begin{itemize}
        \item \textbf{Default branch:} Rama principal
        \item \textbf{Protected branches:} Reglas de protección
        \item \textbf{Deploy keys:} Acceso automatizado
    \end{itemize}
\end{frame}

% ============================================================================
% SECTION 7: PROTECCIÓN DE RAMAS
% ============================================================================

\section{Protección de Ramas}

\begin{frame}{¿Por Qué Proteger Ramas?}
    \vspace{0.3cm}
    
    \begin{orangebox}[El Problema]
        Sin protección, cualquiera puede hacer push directo a \texttt{main} y potencialmente romper producción.
    \end{orangebox}
    
    \vspace{0.4cm}
    \pause
    
    \textbf{Branch Protection Rules permiten:}
    
    \begin{itemize}
        \item Prohibir push directo a main
        \item Requerir aprobaciones antes de merge
        \item Requerir que CI pase antes de merge
        \item Prohibir force push
        \item Requerir firma de commits
    \end{itemize}
\end{frame}

\begin{frame}{Configurar Branch Protection}
    \begin{codebox}[Branch name pattern]
        \ttfamily\color{gludWhite}
        main
    \end{codebox}
    
    \vspace{0.3cm}
    
    \textbf{Opciones recomendadas:}
    \begin{itemize}
        \item[$\checkmark$] Require a pull request before merging
        \item[$\checkmark$] Require status checks to pass
        \item[$\checkmark$] Do not allow bypassing the above settings
        \item[$\checkmark$] Restrict who can push (opcional)
    \end{itemize}
\end{frame}

\begin{frame}{Ejemplo de Reglas por Rama}
    \vspace{0.3cm}
    
    \centering
    \small
    \renewcommand{\arraystretch}{1.3}
    \begin{tabular}{|l|c|c|c|}
        \hline
        \rowcolor{gludOrange}
        \textcolor{white}{\textbf{Rama}} & \textcolor{white}{\textbf{Push Directo}} & \textcolor{white}{\textbf{PR/MR}} & \textcolor{white}{\textbf{CI Requerido}} \\
        \hline
        \rowcolor{gludDarker}
        main & No & Sí & Sí \\
        \hline
        \rowcolor{gludDark}
        develop & Maintainers & Recomendado & Sí \\
        \hline
        \rowcolor{gludDarker}
        feature/* & Developers & No & Opcional \\
        \hline
        \rowcolor{gludDark}
        hotfix/* & Maintainers & Sí & Sí \\
        \hline
    \end{tabular}
    
    \vspace{0.4cm}
    
    {\color{gludLightGray}\scriptsize Esto implementa Git Flow a nivel de permisos del repositorio}
\end{frame}

% ============================================================================
% SECTION 8: WEBHOOKS Y AUTOMATIZACIÓN
% ============================================================================

\section{Webhooks y Automatización}

\begin{frame}{¿Qué son los Webhooks?}
    \vspace{0.3cm}
    
    \begin{orangebox}[Definición]
        Un \highlight{webhook} es una notificación HTTP que GitHub/GitLab envía a un servidor externo cuando ocurre un evento.
    \end{orangebox}
    
    \vspace{0.4cm}
    \pause
    
    \textbf{Eventos comunes:}
    
    \begin{columns}[T]
        \begin{column}{0.48\textwidth}
            \begin{itemize}
                \item Push a una rama
                \item Creación de rama/tag
                \item Pull Request creado
                \item Comentarios
            \end{itemize}
        \end{column}
        \begin{column}{0.48\textwidth}
            \begin{itemize}
                \item Issues
                \item Releases
                \item Merge
                \item Deployment
            \end{itemize}
        \end{column}
    \end{columns}
\end{frame}

\begin{frame}{Casos de Uso de Webhooks}
    \vspace{0.3cm}
    
    \begin{itemize}
        \item \textbf{CI/CD externo:} Trigger de Jenkins, CircleCI
        \item \textbf{Notificaciones:} Slack, Discord, Teams
        \item \textbf{Deployment:} Desplegar automáticamente al push
        \item \textbf{Validaciones:} Verificar código, linting
        \item \textbf{Integraciones:} Jira, Trello, Notion
    \end{itemize}
    
    \vspace{0.4cm}
    \pause
    
    \begin{orangebox}[Configuración]
        {\color{black}Settings → Webhooks → Add webhook\\
        Payload URL, Content type, Secret, Events}
    \end{orangebox}
\end{frame}

\begin{frame}{Deploy Keys}
    \vspace{0.3cm}
    
    \begin{orangebox}[Definición]
        Una \highlight{Deploy Key} es una clave SSH que da acceso a un solo repositorio (no a toda tu cuenta).
    \end{orangebox}
    
    \vspace{0.4cm}
    \pause
    
    \textbf{¿Cuándo usarla?}
    
    \begin{itemize}
        \item Servidores de CI/CD
        \item Scripts automatizados
        \item Acceso de solo lectura para despliegues
        \item Cuando no quieres usar tu clave personal
    \end{itemize}
    
    \vspace{0.3cm}
    
    \textbf{Configuración:} Settings → Deploy keys → Add deploy key
\end{frame}

\begin{frame}{Forks: Contribuir a Proyectos Ajenos}
    \vspace{0.3cm}
    
    \begin{orangebox}[Definición]
        Un \highlight{fork} es una copia de un repositorio ajeno en tu cuenta, donde puedes hacer cambios libremente.
    \end{orangebox}
    
    \vspace{0.4cm}
    \pause
    
    \textbf{Flujo de contribución:}
    
    \begin{enumerate}
        \item Fork del proyecto original
        \item Clone de TU fork
        \item Crear rama, hacer cambios, push
        \item Abrir Pull Request hacia el original
        \item Mantener sincronizado con upstream
    \end{enumerate}
    
    \vspace{0.3cm}
    
    {\color{gludLightGray}\scriptsize Es la forma estándar de contribuir a proyectos open source}
\end{frame}

\begin{frame}{Sincronizar Fork con Upstream}
    \vspace{0.3cm}
    
    \begin{codebox}[Configurar upstream (una vez)]
        \ttfamily\small\color{gludWhite}
        git remote add upstream git@github.com:original/proyecto.git
    \end{codebox}
    
    \vspace{0.3cm}
    \pause
    
    \begin{codebox}[Sincronizar con el original]
        \ttfamily\small\color{gludWhite}
        git fetch upstream\\
        git switch main\\
        git merge upstream/main\\
        git push origin main
    \end{codebox}
    
    \vspace{0.3cm}
    \pause
    
    {\color{gludOrangeLight} GitHub también tiene botón ``Sync fork'' en la interfaz web}
\end{frame}

\begin{frame}{Próxima Clase: Code Review}
    \vspace{0.3cm}
    
    \begin{orangebox}[Clase 7]
        {\color{black}Ya sabemos conectarnos y sincronizar. Ahora aprenderemos el \highlight{arte de revisar código ajeno}.}
    \end{orangebox}
    
    \vspace{0.4cm}
    
    \textbf{En la Clase 7 aprenderemos:}
    \begin{itemize}
        \item Pull Requests / Merge Requests
        \item El proceso de code review
        \item Comentarios constructivos
        \item Aprobaciones y cambios solicitados
        \item Rol rotativo de Maintainer
    \end{itemize}
    
    \vspace{0.4cm}
    
    \centering
    {\color{gludOrangeLight} ¡Preparen sus repositorios con Branch Protection!}
\end{frame}

\quoteslide{Talk is cheap. Show me the code.}{Linus Torvalds}

\thankslide{¡Gracias!\\[0.3cm]{\hspace*{1cm}\Large Ahora están conectados al mundo}}

\end{document}
