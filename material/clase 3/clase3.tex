\documentclass[aspectratio=169,10pt]{beamer}
\usepackage{../theme}

% METADATA

\title{Viajes en el Tiempo}
\subtitle{Historial, Diferencias y Restauración}
\author{GLUD --- Grupo GNU/Linux Universidad Distrital}
\date{Control de Versiones y Desarrollo Colaborativo}

% DOCUMENT

\begin{document}

% Portada
\titleframe[Clase 3 · Viajes en el Tiempo]

\contentsframe

% SECTION 1: INTRODUCCIÓN

\section{Introducción}

\begin{frame}{¿Por qué Viajar en el Tiempo?}
    \vspace{0.3cm}
    
    \begin{orangebox}[El poder de Git]
        Git no solo guarda cambios, te permite \highlight{explorar el pasado}, \highlight{comparar versiones} y \highlight{recuperar estados anteriores}.
    \end{orangebox}
    
    \vspace{0.4cm}
    \pause
    
    \begin{columns}[T]
        \begin{column}{0.48\textwidth}
            \textbf{Escenarios comunes:}
            \begin{itemize}
                \item ¿Cuándo se introdujo este bug?
                \item ¿Qué cambió entre versiones?
                \item ¿Quién modificó esta línea?
                \item Necesito volver a una versión estable
            \end{itemize}
        \end{column}
        \pause
        \begin{column}{0.48\textwidth}
            \textbf{Herramientas para navegar:}
            \begin{itemize}
                \item \texttt{git log} --- Ver historial
                \item \texttt{git diff} --- Comparar cambios
                \item \texttt{git restore} --- Recuperar archivos
                \item \texttt{git tag} --- Marcar versiones
            \end{itemize}
        \end{column}
    \end{columns}
\end{frame}

\quoteslide{Git is a time machine for your code.}{GitHub Guides}

% SECTION 2: HISTORIAL

\section{Explorando el Historial}

\begin{frame}{git log: Tu Máquina del Tiempo}
    \vspace{0.3cm}
    
    \begin{orangebox}[Definición]
        \texttt{git log} muestra el historial completo de commits del repositorio.
    \end{orangebox}
    
    \pause
    
    \begin{codebox}[Uso básico]
        \ttfamily\color{gludWhite}
        git log
    \end{codebox}
    
    \pause
    
    \begin{codebox}[Salida típica]
        \ttfamily\tiny\color{gludWhite}
        commit a3f5b2c8d1e4f6a7b8c9d0e1f2a3b4c5d6e7f8a9\\
        Author: Tu Nombre <tu.email@ejemplo.com>\\
        Date:   Mon Jan 6 10:30:00 2026 -0500\\
        \\
        \hspace{0.5cm}feat(auth): implementar login con JWT
    \end{codebox}
\end{frame}

\begin{frame}{Formatos de git log}
    \vspace{0.3cm}
    
    \begin{columns}[T]
        \begin{column}{0.48\textwidth}
            \begin{codebox}[Una línea por commit]
                \ttfamily\small\color{gludWhite}
                git log -{-}oneline
            \end{codebox}
            
            \pause
            
            \begin{codebox}[Salida compacta]
                \ttfamily\tiny\color{gludWhite}
                a3f5b2c feat(auth): login JWT\\
                f2a3b4c fix(api): error 500\\
                c5d6e7f docs: actualizar README
            \end{codebox}
        \end{column}
        \pause
        \begin{column}{0.48\textwidth}
            \begin{codebox}[Con gráfico]
                \ttfamily\small\color{gludWhite}
                git log -{-}oneline -{-}graph
            \end{codebox}
            
            \pause
            
            \begin{codebox}[Salida visual]
                \ttfamily\tiny\color{gludWhite}
                * a3f5b2c feat(auth): login\\
                * f2a3b4c fix(api): error\\
                * c5d6e7f docs: README\\
                * b4c5d6e Initial commit
            \end{codebox}
        \end{column}
    \end{columns}
    
    \vspace{0.5cm}
    \pause
    
    \centering
    {\color{gludOrangeLight} \texttt{-{-}oneline} es tu mejor aliado para visualizar rápidamente}
\end{frame}

\begin{frame}{Filtrar el Historial}    
    \vspace{-0.25cm}
    \step{1}{Últimos N commits}
    \vspace{-0.25cm}
    \begin{codebox}
        \ttfamily\color{gludWhite}
        git log -n 5 \hspace{1cm}\# Muestra últimos 5 commits
    \end{codebox}
    
    \vspace{-0.25cm}
    \pause
    
    \step{2}{Por autor}
    \vspace{-0.25cm}
    \begin{codebox}
        \ttfamily\color{gludWhite}
        git log -{-}author="Tu Nombre"
    \end{codebox}
    
    \vspace{-0.25cm}
    \pause
    
    \step{3}{Por fecha}
    \vspace{-0.25cm}
    \begin{codebox}
        \ttfamily\small\color{gludWhite}
        git log -{-}since="2026-01-01" -{-}until="2026-01-06"
    \end{codebox}
    
    \vspace{-0.25cm}
    \pause
    
    \step{4}{Por mensaje}
    \vspace{-0.25cm}
    \begin{codebox}
        \ttfamily\color{gludWhite}
        git log -{-}grep="feat"
    \end{codebox}
\end{frame}

\begin{frame}{git log Avanzado}
    \begin{codebox}[Ver archivos modificados]
        \ttfamily\small\color{gludWhite}
        git log -{-}stat
    \end{codebox}
    
    \pause
    
    \begin{codebox}[Ver cambios completos]
        \ttfamily\small\color{gludWhite}
        git log -p
    \end{codebox}
    
    \pause
    
    \begin{codebox}[Formato personalizado]
        \ttfamily\tiny\color{gludWhite}
        git log -{-}pretty=format:"\%h - \%an, \%ar : \%s"
    \end{codebox}
    
    \pause
    
    \begin{codebox}[Salida]
        \ttfamily\tiny\color{gludWhite}
        a3f5b2c - Tu Nombre, 2 hours ago : feat(auth): implementar login\\
        f2a3b4c - Tu Nombre, 1 day ago : fix(api): resolver error 500
    \end{codebox}
\end{frame}

\begin{frame}{Buscar en el Historial}
    \vspace{0.3cm}
    
    \begin{orangebox}[git log -S]
        Busca commits que agregaron o eliminaron una cadena específica.
    \end{orangebox}
    
    \pause
    
    \begin{codebox}[Buscar "calculateTotal"]
        \ttfamily\small\color{gludWhite}
        git log -S "calculateTotal"
    \end{codebox}
    
    \vspace{0.3cm}
    \pause
    
    \begin{codebox}[Ver cambios en archivo específico]
        \ttfamily\small\color{gludWhite}
        git log - - src/Main.java
    \end{codebox}
    
    \vspace{0.3cm}
\end{frame}

% SECTION 3: DIFERENCIAS

\section{Comparando Cambios}

\begin{frame}{git diff: Comparar Versiones}
    \vspace{0.3cm}
    
    \begin{orangebox}[Recordatorio de Clase 1]
        Ya vimos \texttt{git diff} básico. Ahora exploraremos usos avanzados.
    \end{orangebox}
    
    \vspace{0.4cm}
    \pause
    
    \begin{columns}[T]
        \begin{column}{0.48\textwidth}
            \begin{codebox}[Working vs Staged]
                \ttfamily\small\color{gludWhite}
                git diff
            \end{codebox}
            
            {\small\color{gludLightGray} Cambios no agregados}
        \end{column}
        \pause
        \begin{column}{0.48\textwidth}
            \begin{codebox}[Staged vs Committed]
                \ttfamily\small\color{gludWhite}
                git diff -{-}staged
            \end{codebox}
            
            {\small\color{gludLightGray} Cambios en staging}
        \end{column}
    \end{columns}
\end{frame}

\begin{frame}{Comparar entre Commits}
    
    \step{1}{Entre dos commits específicos}
    
    \begin{codebox}
        \ttfamily\small\color{gludWhite}
        git diff a3f5b2c f2a3b4c
    \end{codebox}
    
    \pause
    
    \step{2}{Commit actual vs anterior}
    
    \begin{codebox}
        \ttfamily\small\color{gludWhite}
        git diff HEAD\textasciitilde{}1 HEAD
    \end{codebox}
    
    \pause
    
    \step{3}{Últimos N commits}
    
    \begin{codebox}
        \ttfamily\small\color{gludWhite}
        git diff HEAD\textasciitilde{}3 HEAD
    \end{codebox}
\end{frame}

\begin{frame}{Referencias Relativas}
    \vspace{0.3cm}
    
    \begin{orangebox}[Definición]
        Las \highlight{referencias relativas} permiten navegar el historial desde un punto.
    \end{orangebox}
    
    \vspace{0.4cm}
    \pause
    
    \begin{columns}[T]
        \begin{column}{0.48\textwidth}
            \textbf{Notación \textasciitilde{}}
            \begin{itemize}
                \item \texttt{HEAD\textasciitilde{}1} --- 1 commit atrás
                \item \texttt{HEAD\textasciitilde{}2} --- 2 commits atrás
                \item \texttt{HEAD\textasciitilde{}5} --- 5 commits atrás
            \end{itemize}
        \end{column}
        \pause
        \begin{column}{0.48\textwidth}
            \textbf{Notación \^{}}
            \begin{itemize}
                \item \texttt{HEAD\^{}} --- Padre directo
                \item \texttt{HEAD\^{}\^{}} --- Abuelo
                \item \texttt{HEAD\^{}2} --- Segundo padre (merge)
            \end{itemize}
        \end{column}
    \end{columns}
    
    \vspace{0.5cm}
    \pause
    
    \centering
    {\color{gludLightGray} Para linearidad simple, \texttt{\textasciitilde{}} es más claro}
\end{frame}

\begin{frame}{diff en Archivos Específicos}
    \vspace{0.3cm}
    
    \begin{codebox}[Diferencias en un archivo]
        \ttfamily\small\color{gludWhite}
        git diff HEAD\textasciitilde{}1 HEAD -- src/Main.java
    \end{codebox}
    
    \pause
    
    \begin{codebox}[Salida típica]
        \ttfamily\tiny\color{gludWhite}
        diff --git a/src/Main.java b/src/Main.java\\
        index a3f5b2c..f2a3b4c 100644\\
        --- a/src/Main.java\\
        +++ b/src/Main.java\\
        @@ -10,7 +10,7 @@ public class Main \{\\
        \hspace{0.5cm}public static void main(String[] args) \{\\
        -\hspace{1cm}System.out.println("Hola Mundo");\\
        +\hspace{1cm}System.out.println("Hola GLUD");\\
        \hspace{0.5cm}\}
    \end{codebox}
    
    \vspace{0.3cm}
    \pause
    
    \centering
    {\color{gludLightGray} \texttt{-} línea eliminada, \texttt{+} línea agregada}
\end{frame}

\begin{frame}{Herramientas Visuales para diff}
    \vspace{0.3cm}
    
    \begin{orangebox}[git difftool]
        Lanza una herramienta gráfica para visualizar diferencias.
    \end{orangebox}
    
    \pause
    
    \begin{codebox}[Configurar difftool]
        \ttfamily\small\color{gludWhite}
        git config -{-}global diff.tool vimdiff\\
        \# Alternativas: meld, kdiff3, vscode
    \end{codebox}
    
    \pause
    
    \begin{codebox}[Usar difftool]
        \ttfamily\small\color{gludWhite}
        git difftool HEAD\textasciitilde{}1 HEAD
    \end{codebox}
    \pause
    
    \centering
    {\color{gludOrangeLight} Recomendado para cambios complejos con múltiples archivos}
\end{frame}

% SECTION 4: RESTAURACIÓN

\section{Restaurar y Recuperar}

\begin{frame}{git restore: Recuperar Archivos}
    \vspace{0.3cm}
    
    \begin{orangebox}[Definición]
        \texttt{git restore} recupera archivos desde el staging o desde commits anteriores.
    \end{orangebox}
    
    \pause
    
    \begin{codebox}[Descartar cambios en working directory]
        \ttfamily\small\color{gludWhite}
        git restore archivo.txt
    \end{codebox}
    
    \pause
    
    \begin{alertbox}[Advertencia]
        \textbf{¡Esto elimina tus cambios locales permanentemente!}
    \end{alertbox}
    
    \vspace{0.3cm}
    \pause
    
    \centering
    {\color{gludLightGray} Úsalo solo cuando estés seguro de descartar cambios}
\end{frame}

\begin{frame}{Quitar Archivos del Staging}
    \vspace{0.3cm}
    
    \begin{codebox}[Escenario]
        \ttfamily\small\color{gludWhite}
        git add archivo.txt\\
        \# ¡Ups! No quería agregarlo todavía
    \end{codebox}
    
    \pause
    
    \begin{codebox}[Solución]
        \ttfamily\small\color{gludWhite}
        git restore -{-}staged archivo.txt
    \end{codebox}
    
    \vspace{0.4cm}
    \pause
    
    \begin{darkbox}[Resultado]
        El archivo vuelve a Working Directory. Tus cambios \textbf{se mantienen}.
    \end{darkbox}
    
    \vspace{0.3cm}
    \pause
    
    \centering
    {\color{gludOrangeLight} Útil para reorganizar qué va en cada commit}
\end{frame}

\begin{frame}{Restaurar desde Commit Específico}
    \vspace{0.3cm}
    
    \begin{codebox}[Recuperar versión antigua]
        \ttfamily\small\color{gludWhite}
        git restore -{-}source=HEAD\textasciitilde{}3 archivo.txt
    \end{codebox}
    
    \pause
    
    \begin{codebox}[Con hash de commit]
        \ttfamily\small\color{gludWhite}
        git restore -{-}source=a3f5b2c archivo.txt
    \end{codebox}
    
    \vspace{0.4cm}
    \pause
    
    \begin{darkbox}[Uso típico]
        Recuperar un archivo que funcionaba 3 commits atrás sin afectar otros archivos.
    \end{darkbox}
\end{frame}

\begin{frame}{git checkout vs git restore}
    \vspace{0.3cm}
    
    \begin{orangebox}[Historia]
        \texttt{git checkout} hacía muchas cosas. Git 2.23+ lo dividió en comandos especializados.
    \end{orangebox}
    
    \vspace{0.4cm}
    \pause
    
    \begin{columns}[T]
        \begin{column}{0.48\textwidth}
            \textbf{Forma antigua:}
            \begin{codebox}
                \ttfamily\tiny\color{gludWhite}
                git checkout archivo.txt\\
                git checkout -{-} archivo.txt\\
                git checkout HEAD archivo.txt
            \end{codebox}
        \end{column}
        \pause
        \begin{column}{0.48\textwidth}
            \textbf{Forma moderna:}
            \begin{codebox}
                \ttfamily\tiny\color{gludWhite}
                git restore archivo.txt\\
                git restore archivo.txt\\
                git restore -{-}source=HEAD archivo.txt
            \end{codebox}
        \end{column}
    \end{columns}
    
    \vspace{0.5cm}
    \pause
    
    \centering
    {\color{gludLightGray} Usa \texttt{git restore}, es más claro y específico}
\end{frame}

\begin{frame}{Recuperar Archivos Eliminados}
    \vspace{0.3cm}
    
    \begin{codebox}[Escenario: Eliminaste un archivo]
        \ttfamily\small\color{gludWhite}
        rm src/Config.java\\
        git status \hspace{1cm}\# Muestra: deleted: src/Config.java
    \end{codebox}
    
    \pause
    
    \begin{codebox}[Recuperar desde último commit]
        \ttfamily\small\color{gludWhite}
        git restore src/Config.java
    \end{codebox}
    
    \vspace{0.4cm}
    \pause
    
    \begin{darkbox}[Resultado]
        El archivo reaparece con el contenido del último commit.
    \end{darkbox}
\end{frame}

% SECTION 5: ETIQUETAS

\section{Etiquetas y Versiones}

\begin{frame}{¿Qué son las Etiquetas (Tags)?}
    \vspace{0.3cm}
    
    \begin{orangebox}[Definición]
        Un \highlight{tag} es un marcador permanente que apunta a un commit específico, típicamente usado para versiones.
    \end{orangebox}
    
    \vspace{0.4cm}
    \pause
    
    \begin{columns}[T]
        \begin{column}{0.48\textwidth}
            \textbf{Usos comunes:}
            \begin{itemize}
                \item Marcar releases (\texttt{v1.0.0})
                \item Identificar versiones estables
                \item Puntos de referencia importantes
                \item Facilitar rollbacks
            \end{itemize}
        \end{column}
        \pause
        \begin{column}{0.48\textwidth}
            \textbf{Tipos de tags:}
            \begin{itemize}
                \item \textbf{Ligeros} --- Solo un puntero
                \item \textbf{Anotados} --- Con metadata completa
            \end{itemize}
            
            \vspace{0.3cm}
            
            {\color{gludOrangeLight} Usa tags anotados para releases}
        \end{column}
    \end{columns}
\end{frame}

\begin{frame}{Crear Tags}
    \step{1}{Tag ligero}
    
    \begin{codebox}
        \ttfamily\color{gludWhite}
        git tag v0.1.0-alpha
    \end{codebox}
    
    \pause
    
    \step{2}{Tag anotado (recomendado)}
    
    \begin{codebox}
        \ttfamily\small\color{gludWhite}
        git tag -a v0.1.0-alpha -m "Primera versión alpha"
    \end{codebox}
    
    \pause
    
    \step{3}{Tag en commit específico}
    
    \begin{codebox}
        \ttfamily\small\color{gludWhite}
        git tag -a v0.0.9 a3f5b2c -m "Versión de prueba"
    \end{codebox}
\end{frame}

\begin{frame}{Listar y Ver Tags}
    \vspace{-0.25cm}
    \begin{codebox}[Listar todos los tags]
        \ttfamily\color{gludWhite}
        git tag
    \end{codebox}
    
    \pause
    \begin{codebox}[Buscar tags por patrón]
        \ttfamily\color{gludWhite}
        git tag -l "v0.1.*"
    \end{codebox}
    
    \pause
    
    \begin{codebox}[Ver información de tag anotado]
        \ttfamily\color{gludWhite}
        git show v0.1.0-alpha
    \end{codebox}
    
    \pause
    
    \begin{codebox}[Salida]
        \ttfamily\tiny\color{gludWhite}
        tag v0.1.0-alpha\\
        Tagger: Tu Nombre <tu.email@ejemplo.com>\\
        Date:   Mon Jan 6 15:00:00 2026 -0500\\
        \\
        Primera versión alpha
    \end{codebox}
\end{frame}

\begin{frame}{Versionamiento Semántico (SemVer)}
    \vspace{0.3cm}
    
    \begin{orangebox}[Formato]
        \texttt{vMAYOR.MENOR.PARCHE-sufijo}
    \end{orangebox}
    
    \vspace{0.4cm}
    \pause
    
    \begin{columns}[T]
        \begin{column}{0.32\textwidth}
            \centering
            \textbf{MAYOR}
            
            {\small Cambios incompatibles}
            
            \vspace{0.2cm}
            
            {\tiny \texttt{1.x.x} → \texttt{2.0.0}}
        \end{column}
        \begin{column}{0.32\textwidth}
            \centering
            \textbf{MENOR}
            
            {\small Nuevas funcionalidades}
            
            \vspace{0.2cm}
            
            {\tiny \texttt{1.2.x} → \texttt{1.3.0}}
        \end{column}
        \begin{column}{0.32\textwidth}
            \centering
            \textbf{PARCHE}
            
            {\small Correcciones de bugs}
            
            \vspace{0.2cm}
            
            {\tiny \texttt{1.2.3} → \texttt{1.2.4}}
        \end{column}
    \end{columns}
    
    \vspace{0.5cm}
    \pause
    
    \centering
    
    \textbf{Sufijos:} \texttt{-alpha}, \texttt{-beta}, \texttt{-rc1}, \texttt{-rc2}
    
    \vspace{0.2cm}
    
    {\color{gludLightGray} Ejemplos: \texttt{v0.1.0-alpha}, \texttt{v1.0.0-beta}, \texttt{v2.3.1}}
\end{frame}

\begin{frame}{Crear Versión v0.1.0-alpha}
    \vspace{0.3cm}
    
    \begin{codebox}[Práctica guiada]
        \ttfamily\small\color{gludWhite}
        \# Asegúrate de estar en un estado limpio\\
        git status\\
        \\
        \# Crea el tag anotado\\
        git tag -a v0.1.0-alpha -m "Release: Primera versión alpha"\\
        \\
        \# Verifica\\
        git tag\\
        git show v0.1.0-alpha
    \end{codebox}
    
    \vspace{0.4cm}
    \pause
    
    \centering
    {\color{gludOrangeLight} Este tag marca tu primer hito en el proyecto}
\end{frame}

\begin{frame}{Compartir Tags}
    \vspace{0.3cm}
    
    \begin{alertbox}[Importante]
        Por defecto, \texttt{git push} \textbf{NO} envía tags al remoto.
    \end{alertbox}
    
    \pause
    
    \begin{codebox}[Enviar un tag específico]
        \ttfamily\small\color{gludWhite}
        git push origin v0.1.0-alpha
    \end{codebox}
    
    \pause
    
    \begin{codebox}[Enviar todos los tags]
        \ttfamily\small\color{gludWhite}
        git push origin -{-}tags
    \end{codebox}

    \pause
    
    \centering
    {\color{gludLightGray} En GitHub/GitLab, los tags aparecen en la sección "Releases"}
\end{frame}

\begin{frame}{Eliminar Tags}
    \vspace{0.3cm}
    
    \begin{codebox}[Eliminar tag local]
        \ttfamily\color{gludWhite}
        git tag -d v0.1.0-alpha
    \end{codebox}
    
    \pause
    
    \begin{codebox}[Eliminar tag remoto]
        \ttfamily\small\color{gludWhite}
        git push origin -{-}delete v0.1.0-alpha
    \end{codebox}
    
    \vspace{0.4cm}
    \pause
    
    \begin{alertbox}[Advertencia]
        Ten cuidado al eliminar tags públicos. Otros pueden estar usándolos.
    \end{alertbox}
\end{frame}

\begin{frame}{Navegar a un Tag}
    \vspace{0.3cm}
    
    \begin{codebox}[Ver estado en un tag]
        \ttfamily\small\color{gludWhite}
        git checkout v0.1.0-alpha
    \end{codebox}
    
    \pause
    
    \begin{alertbox}[Estado "detached HEAD"]
        Estarás en modo lectura. Para trabajar, crea una rama desde el tag.
    \end{alertbox}
    
    \pause
    
    \begin{codebox}[Crear rama desde tag]
        \ttfamily\small\color{gludWhite}
        git checkout -b bugfix-v0.1 v0.1.0-alpha
    \end{codebox}
    
    \vspace{0.3cm}
    \pause
    
    \centering
    {\color{gludLightGray} Útil para crear hotfixes sobre versiones específicas}
\end{frame}


% SECTION 6: BUENAS PRÁCTICAS

\section{Buenas Prácticas}

\begin{frame}{Cuándo Usar Cada Comando}
    \begin{columns}[T]
        \begin{column}{0.48\textwidth}
            \begin{orangebox}[git log]
                \begin{itemize}
                    \item {\color{black}Revisar historial
                    \item Buscar cuándo cambió algo
                    \item Encontrar autor de cambios
                    \item Auditar el proyecto}
                \end{itemize}
            \end{orangebox}
            
            
            \begin{orangebox}[git diff]
                \begin{itemize}
                    \item {\color{black}Revisar antes de commit
                    \item Comparar versiones
                    \item Entender cambios ajenos}
                \end{itemize}
            \end{orangebox}
        \end{column}
        \pause
        \begin{column}{0.48\textwidth}
            \begin{darkbox}[git restore]
                \begin{itemize}
                    \item Descartar cambios locales
                    \item Quitar archivos de staging
                    \item Recuperar archivos eliminados
                    \item Probar código antiguo
                \end{itemize}
            \end{darkbox}
            
            \begin{darkbox}[git tag]
                \begin{itemize}
                    \item Marcar releases
                    \item Identificar versiones estables
                    \item Crear puntos de restauración
                \end{itemize}
            \end{darkbox}
        \end{column}
    \end{columns}
\end{frame}

\begin{frame}{Flujo de Trabajo Recomendado}
    \vspace{0.3cm}
    
    \step{1}{Antes de cada commit}
    
    \begin{codebox}
        \ttfamily\small\color{gludWhite}
        git status \hspace{1cm}\# ¿Qué cambió?\\
        git diff \hspace{1.3cm}\# ¿Cómo cambió?\\
        git add archivo \hspace{0.5cm}\# Agregar selectivamente
    \end{codebox}
    
    \pause
    
    \step{2}{Después de varios commits}
    
    \begin{codebox}
        \ttfamily\small\color{gludWhite}
        git log -{-}oneline \hspace{0.5cm}\# Revisar historial\\
        git diff HEAD\textasciitilde{}5 HEAD \hspace{0.1cm}\# ¿Qué cambió en total?
    \end{codebox}
\end{frame}

\begin{frame}{Flujo de Trabajo Recomendado}
    \vspace{0.3cm}
    
    \step{3}{Al alcanzar un hito}
    
    \begin{codebox}
        \ttfamily\small\color{gludWhite}
        git tag -a v0.1.0-alpha -m "Primera versión funcional"
    \end{codebox}
\end{frame}

\begin{frame}{Errores Comunes}
    \vspace{0.3cm}
    
    \begin{alertbox}[\ding{55} No hagas esto]
        \begin{itemize}
            \item {\color{black}Usar \texttt{git restore} sin verificar cambios
            \item Eliminar tags públicos sin comunicar
            \item Ignorar \texttt{git diff} antes de commit
            \item Crear tags sin mensajes descriptivos
            \item No usar versionamiento semántico}
        \end{itemize}
    \end{alertbox}
    
\end{frame}
\begin{frame}{Errores Comunes}
    \begin{darkbox}[\ding{51} Haz esto]
        \begin{itemize}
            \item Siempre revisa \texttt{git diff} antes de commit
            \item Usa tags anotados para releases
            \item Sigue SemVer consistentemente
            \item Documenta cambios importantes en tags
        \end{itemize}
    \end{darkbox}
\end{frame}

\quoteslide{The best code is well-versioned code.}{GitHub Guides}

% SECTION 7: RESUMEN

\section{Resumen}

\begin{frame}{Lo que Aprendimos}
    \vspace{0.3cm}
    
    \begin{columns}[T]
        \begin{column}{0.48\textwidth}
            \textbf{Historial y Navegación}
            \begin{itemize}
                \item \texttt{git log} y sus variantes
                \item Filtros por autor, fecha, mensaje
                \item Referencias relativas (\texttt{HEAD\textasciitilde{}N})
                \item Buscar cambios específicos
            \end{itemize}
            
            \vspace{0.4cm}
            
            \textbf{Comparación}
            \begin{itemize}
                \item \texttt{git diff} entre commits
                \item Diferencias en archivos específicos
                \item Herramientas visuales (difftool)
            \end{itemize}
        \end{column}
        \pause
        \begin{column}{0.48\textwidth}
            \textbf{Restauración}
            \begin{itemize}
                \item \texttt{git restore} para working directory
                \item \texttt{git restore -{-}staged} para staging
                \item Recuperar desde commits específicos
                \item Diferencia con \texttt{git checkout}
            \end{itemize}
            
            \vspace{0.4cm}
            
            \textbf{Versionamiento}
            \begin{itemize}
                \item Tags ligeros y anotados
                \item Versionamiento semántico
                \item Compartir y eliminar tags
                \item Crear \texttt{v0.1.0-alpha}
            \end{itemize}
        \end{column}
    \end{columns}
\end{frame}

\begin{frame}{Comandos Clave}    
    \begin{codebox}
        \ttfamily\small\color{gludWhite}
        \# Historial\\
        git log -{-}oneline -{-}graph\\
        git log -{-}author="Nombre" -{-}since="2026-01-01"\\
        \\
        \# Diferencias\\
        git diff HEAD\textasciitilde{}1 HEAD\\
        git diff a3f5b2c f2a3b4c -{-} archivo.txt\\
        \\
        \# Restauración\\
        git restore archivo.txt\\
        git restore -{-}staged archivo.txt\\
        \\
        \# Tags\\
        git tag -a v0.1.0-alpha -m "Primera versión"\\
        git push origin -{-}tags
    \end{codebox}
\end{frame}

\begin{frame}{Recursos}
    \vspace{0.5cm}
    
    \begin{columns}[T]
        \begin{column}{0.48\textwidth}
            \textbf{Git:}
            \begin{itemize}
                \item \url{https://git-scm.com/docs/git-log}
                \item \url{https://git-scm.com/docs/git-diff}
                \item \url{https://git-scm.com/docs/git-restore}
                \item \url{https://git-scm.com/docs/git-tag}
            \end{itemize}
        \end{column}
        \begin{column}{0.48\textwidth}
            \textbf{Versionamiento:}
            \begin{itemize}
                \item \url{https://semver.org/}
                \item Conventional Commits
                \item Git Visualization Tools
                \item GitHub Releases Guide
            \end{itemize}
        \end{column}
    \end{columns}
    
    \vspace{0.8cm}
    
    \textbf{Práctica interactiva:}
    \begin{itemize}
        \item \url{https://learngitbranching.js.org/}
        \item Git Immersion: \url{https://gitimmersion.com/}
    \end{itemize}
\end{frame}

% Next class preview

\begin{frame}[plain]
\setbeamercolor{background canvas}{bg=gludPeach}
\vspace{1.5cm}
\centering

{\Huge\color{gludOrange}\bfseries Próxima Clase}

\vspace{0.5cm}

{\Large\color{gludDark} El Multiverso: Branching}

\vspace{1cm}

{\large\color{gludWhite} Concepto de punteros y referencias. Ramas feature/* y hotfix/*. Por qué nunca trabajar directo en main.}
\end{frame}

% Despedida

\thankslide{¡Gracias!\\[0.3cm]{\hspace*{1cm}\Large Nos vemos en la próxima clase}}

\end{document}
